\chapter{C\'odigo de Conducta}\label{s:conduct}

Con el objetivo de fomentar un ambiente abierto y amigable, las personas 
a cargo, nos comprometemos a hacer de la participaci\'on en nuestro proyecto 
y nuestra comunidad una experiencia libre de acoso para todas las personas, 
independientemente de su edad, tama\~no corporal, discapacidad, etnia, 
identidad y expresi\'on de g\'enero, nivel de experiencia, educaci\'on, 
nivel socioecon\'omico, nacionalidad, apariencia personal, raza, 
religi\'on o identidad y orientaci\'on sexual.

\section*{Nuestros est\'andares}

Ejemplos de comportamiento que contribuyen a crear un ambiente positivo 
para nuestra comunidad:

\begin{itemize}
\item
  utilizar un lenguaje amigable e inclusivo, 
\item
  respetar diferentes opiniones, puntos de vista y experiencias,
\item
  aceptar adecuadamente la cr\'itica constructiva,
\item
  enfoc\'andose en lo que es mejor para la comunidad y
\item
  mostrando empat\'ia hacia otros miembros de la comunidad.
\end{itemize}

\noindent
Ejemplos de comportamiento inaceptable:

\begin{itemize}
\item
  el uso de lenguaje o im\'agenes sexualizadas como tambi\'en 
  atenci\'on o avances sexuales no deseados,
\item
  comentarios despectivos (trolling), insultantes y ataques personales o pol\'iticos,
\item
  cualquier tipo de acoso en p\'ublico o privado,
\item
  Publicar informaci\'on privada de otras personas, tales como direcciones 
  f\'isicas o de correo electr\'onico, sin su permiso expl\'icito
\item
  Otras conductas que puedan ser razonablemente consideradas 
  como inapropiadas en un entorno profesional
\end{itemize}

\section*{Nuestras responsabilidades}

Las personas encargadas el proyecto somos responsables de aclarar los est\'andares de
comportamiento aceptable y se espera que tomemos medidas de acci\'on correctiva 
apropiadas y justas en respuesta a cualquier caso de comportamiento que 
consideremos inaceptable.

Las personas encargadas del proyecto tienen el derecho y la responsabilidad de 
eliminar, editar o rechazar comentarios, commits, c\'odigo, ediciones wiki, issues y otras
contribuciones que no est\'an alineadas con este c\'odigo de conducta. Tambi\'en pueden 
prohibir la participaci\'on temporal o permanente de cualquier persona por comportamientos 
que consideremos inapropiados, amenazantes, ofensivos o da\~ninos.

\section*{Alcance}

Este c\'odigo de conducta se aplica dentro de los espacios del proyecto 
y en espacios p\'ublicos, cuando una persona representa al proyecto o a
la comunidad. Ejemplos de representaci\'on del proyecto o la comunidad incluyen
el uso de una direcci\'on de correo electr\'onico oficial del proyecto, 
realizar publicaciones a trav\'es de una cuenta oficial de redes sociales, 
o actuar como representante oficial en cualquier tipo de evento. 
La representaci\'on del proyecto puede ser aclarada y definida en m\'as
detalles por las personas encargadas.

\section*{Aplicaci\'on}

Los casos de comportamiento abusivo, acosador o inaceptable 
pueden ser informados enviando un correo electr\'onico a \texttt{gvwilson@third-bit.com}. 
Todas las quejas ser\'an revisadas e investigadas y dar\'an como resultado 
la respuesta que se considere necesaria y apropiada a las circunstancias.
El equipo encargado del proyecto est\'a obligado a mantener la privacidad de 
quienes reporten incidentes.  Se pueden publicar por separado m\'as detalles 
de pol\'iticas de aplicaci\'on espec\'ificas.

Las personas encargadas del proyecto que no cumplan o hagan cumplir 
de buena fe este c\'odigo de conducta pueden enfrentar repercusiones 
temporales o permanentes determinadas por el resto del equipo encargado 
del proyecto.

\section*{Atribuci\'on}

Este c\'odigo de conducta es una adaptaci\'on del 
\hreffoot{https://www.contributor-covenant.org}{Contributor Covenant} version 1.4.
