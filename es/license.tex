\chapter{Licencia}\label{s:license}

{\setlength{\parindent}{0em}

\emph{
  Este es un resumen de lectura sencilla para personas (y no un sustituto) de la licencia.
  Por favor mira \url{https://creativecommons.org/licenses/by-nc/4.0/legalcode} para el texto legal completo.
}

\vspace{\baselineskip}

\noindent
Este trabajo est\'a licenciado bajo 
\hreffoot{https://creativecommons.org/licenses/by-nc/4.0/}{Creative Commons Atribuci\'on – No Comercial 4.0} 
(CC-BY-NC-4.0).\\

\noindent
\textbf{Eres libre de:}

\begin{itemize}
\item
  \textbf{Compartir}---copiar y redistribuir el material en cualquier medio o
  formato
\item
  \textbf{Adaptar}---reacomodar, transformar y construir sobre el material.
\end{itemize}

El licenciante no puede revocar estas libertades mientras sigas los
t\'erminos de la licencia.

\vspace{\baselineskip}

\textbf{Bajo los siguientes t\'erminos:}

\begin{itemize}
\item
  \textbf{Atribuci\'on}---Debes dar el cr\'edito apropiado, proporcionar un enlace
  a la licencia e indicar si se hicieron cambios. Puedes hacerlo de manera 
  razonable, pero no de manera que sugiera que el licenciante te respalda 
  a ti o al uso que haces del material. \\
\item
  \textbf{No Comercial}---No puedes utilizar el material con fines comerciales.
\end{itemize}

\textbf{Sin restricciones adicionales}---No puedes aplicar t\'erminos legales o
medidas tecnol\'ogicas que restringen legalmente a otros de hacer cualquier cosa
que la licencia permita.

\vspace{\baselineskip}

\textbf{Avisos:}

\begin{itemize}

\item
  No tienes que cumplir con la licencia para elementos del 
  material de dominio p\'ublico o donde su uso est\'e permitido 
  por una excepci\'on o limitaci\'on aplicable.  

\item
  No se otorgan garant\'ias. Es posible que la licencia no te otorgue 
  todos los permisos necesarios para tu intenci\'on de uso. Por ejemplo, otros 
  derechos como publicidad, privacidad o derechos morales pueden limitar 
  la forma en la que puedes usar el material.

\end{itemize}
}
