\chapter{Construyendo una comunidad de práctica}\label{s:community}

No tienes que arreglar todos los males de la sociedad para enseñar programación,
pero \emph{si} te tienes que involucrar en
lo que sucede fuera de tu clase si quieres que las personas aprendan.
Esto se aplica tanto a las personas que enseñan como a las que aprenden:
muchos docentes free-range comienzan como voluntarios o como trabajadores a medio tiempo
y deben hacer malabares con sus clases y muchos otros compromisos.
Lo que sucede fuera del aula es tan importante para su éxito 
como lo es para sus estudiantes,
así que la mejor manera de ayudar a ambos es fomentar una comunidad de enseñanza.


\begin{aside}{Finlandia y por qué no}
Las escuelas de Finlandia se encuentran entre las más exitosas del mundo,
  pero como Anu Partanen\index{Partanen, Anu}
  \hreffoot{https://www.theatlantic.com/national/archive/2011/12/what-americans-keep-ignoring-about-finlands-school-success/250564/}{señaló},
  no han tenido éxito solas.
  Los intentos de otros países por adoptar métodos de enseñanza finlandeses están condenados al fracaso
  a menos que esos países también garanticen que los niños (y sus padres) estén seguros,
  bien alimentados,
  y tratados justamente por las cortes~\cite{Sahl2015, Wilk2011}.
  Esto no es ninguna sorpresa dado lo que sabemos sobre la importancia de la motivación para el aprendizaje (\chapref{s:motivation}):
  todas las personas lo harán peor si creen que el sistema es impredecible, injusto o indiferente.
\end{aside}

Un marco para pensar sobre las comunidades de enseñanza es el \gref{g:situado-learning}{aprendizaje situado},
que se centra en cómo una \gref{g:legitimate-peripheral-participation}{participación periférica legítima}
lleva a las personas a convertirse en miembros de
una \gref{g:community-of-practice}{comunidad de práctica}~\cite{Weng2015}.
Analizando esos términos,
una comunidad de práctica es un grupo de personas unidas por su interés en alguna actividad,
como tejer o la física de partículas.
La participación periférica legítima significa realizar tareas simples y de bajo riesgo
que la comunidad reconoce como contribuciones válidas:
hacer tu primera bufanda,
llenar sobres durante una campaña electoral,
o revisar documentación de software de código abierto.

Situated learning focuses on the transition from being a newcomer
to being accepted as a peer by those who are already community members.
This typically means starting with simple tasks and tools,
then doing similar tasks with more complex tools,
and finally tackling the same work as advanced practitioners.
For example,
children learning music may start by playing nursery rhymes on a recorder or ukulele,
then play other simple songs on a trumpet or saxophone in a band,
and finally start exploring their own musical tastes.
Common ways to support this progression include:

El aprendizaje situado se centra en la transición de ser un recién llegado
a ser aceptado como un compañero por aquellos que ya son miembros de la comunidad.
Esto generalmente significa comenzar con tareas y herramientas simples,
luego hacer tareas similares con herramientas más complejas,
y finalmente abordar el mismo trabajo que los practicantes avanzados.
Por ejemplo,
los niños que aprenden música pueden comenzar tocando canciones infantiles en una grabadora o un ukelele,
luego tocan otras canciones simples en una trompeta o saxofón en una banda,
y finalmente comienzan a explorar sus propios gustos musicales.
Las formas comunes de apoyar esta progresión incluyen:

\newpage
\begin{description}

\item[Resolución de problemas:]
  ``No puedo avanzar --- ¿podemos trabajar en el diseño de esta lección juntos? ''
\item[Requests for information:]
  ``¿Cuál es la contraseña para el administrador de la lista de correo? ''
\item[Búsqueda de experiencia:]
  ``¿Alguien ha tenido un alumno con discapacidad para leer? ''
\item[Compartir recursos:]
  ``El año pasado armé un sitio web para una clase que puedes usar como punto de partida ''.
\item[Coordinación:]
  ``¿Podemos hacer nuestros pedidos de camisetas juntos para obtener un descuento? ''
\item[Construir un argumento:]
  ``Será más fácil convencer a mi jefe para que haga cambios si sé cómo otros campistas hacen esto ''.
\item[Documentar proyectos:]
  ``Hemos tenido este problema cinco veces ahora. Vamos a escribirlo de una vez por todas ''.
\item[Mapeo del conocimiento:]
  ``¿Qué otros grupos están haciendo cosas como esta en vecindarios o ciudades cercanas? ''
\item[Visitas:]
  ``¿Podemos venir a ver su programa extracurricular? Necesitamos establecer uno en nuestra ciudad ''.
\end{description}

\hreffoot{https://www.feverbee.com/types-of-community-and-activity-within-the-community/}{En términos generales},
Una comunidad de práctica puede ser:

\begin{description}
\item[Comunidad de acción:]
  personas enfocadas en un objetivo compartido,
  como conseguir que un candidato sea elegido.

\item[Comunidad de interés:]
  los miembros se unen por un problema compartido,
  como tratar una enfermedad a largo plazo.

\item[Comunidad de interés:]
  enfocado en un amor compartido por algo como el backgammon o tejer.

\item[Comunidad de lugar:]
  personas que viven o trabajan juntas.

\end{description}
  
La mayoría de las comunidades son mezclas de estos tipos,
como las personas en Toronto a las que les gusta enseñar tecnología.
El enfoque de una comunidad también puede cambiar con el tiempo:
por ejemplo,
un grupo de apoyo para personas que padecen depresión (comunidad de interés)
puede decidir recaudar fondos para mantener una línea de ayuda (comunidad de acción).
Mantener el funcionamiento la línea de ayuda puede convertirse en el foco del grupo (comunidad de interés).

\begin{aside}{Sopa, luego himnos}
  Los manifiestos son divertidos de escribir,
  pero la mayoría de las personas se unen a una comunidad de voluntarios para ayudar y ser ayudados
  en lugar de discutir sobre la redacción de una declaración de visión\footnote{
    Las personas que prefieren lo último a menudo están \emph{solo} interesadas en discutir{\ldots}}.
  Por lo tanto, debes centrarte en
  qué personas pueden crear lo que otros miembros de la comunidad usarán de inmediato.
  Una vez que tu organización demuestre que puede lograr cosas pequeñas,
  la gente estará más segura de que vale la pena ayudarte con proyectos más grandes.
  Es el momento de preocuparse por definir los valores que guiarán a sus miembros.
\end{aside}

\seclbl{Aprende, luego haz}{s:community-learn-then-do}

El primer paso para construir una comunidad es decidir si deberías construirla,
o si sería más efectivo unirte a una organización existente.
Miles de grupos ya están enseñando habilidades tecnológicas a las personas,
desde los \hreffoot{http://www.4-h-canada.ca/}{4-H Club}
y los \hreffoot{https://www.frontiercollege.ca/}{programas de alfabetización}
hasta organizaciones sin fines de lucro que te inician en la programación como
\hreffoot{http://www.blackgirlscode.com/}{Black Girls Code}
y \hreffoot{http://bridgeschool.io/}{Bridge}.
Unirse a un grupo existente te dará ventaja en la enseñanza,
un conjunto inmediato de colegas,
y una oportunidad de aprender más sobre cómo manejar las cosas;
Con suerte,
ir aprendiendo esas habilidades mientras haces una contribución inmediata
será más importante que poder decir que
sos el fundador o líder de algo nuevo.

Ya sea que te unas a un grupo existente o inicies uno propio,
serás más efectivo si lees un poco sobre como organizar una comunidad.
\cite{Alin1989,Lake2018} probablemente es el trabajo más conocido sobre organización de grupos de base,
mientras que ~\cite{Brow2007,Midw2010,Lake2018} son manuales prácticos basados ​​en décadas de experiencia.
Si quieres leer más profundamente,
\cite{Adam1975} es la historia de la Highlander Folk School,
cuyo enfoque ha sido emulado por muchos grupos exitosos,
mientras que ~\cite{Spal2014} es una guía para enseñar a adultos
escrita por alguien con profundas raíces personales en la organización
y \hreffoot{https://www.nonprofitready.org/}{NonprofitReady.org}
ofrece capacitación profesional gratuita.

\seclbl{Cuatro pasos}{s:community-four-steps}

Todas las personas que se involucren con tu organización
(incluyéndote)
pasa por cuatro fases:
reclutamiento, incorporación, retención y retiro.
No necesitas preocuparte por este ciclo cuando estés comenzando,
pero vale la pena pensar en este tema
tan pronto como más de un puñado de personas no fundadoras estén involucradas.

El primer paso es reclutar voluntarias y voluntarios.\index{recruitment (of members)}
Tu estrategia de marketing debería ayudarte con esto haciendo que tu organización sea localizable\index{findability!of organizations}
y que tu misión y valor sean claros
para las personas que quieran involucrarse

(\chapref{s:outreach}).
Comparte historias que ejemplifiquen el tipo de ayuda que desea
así como historias sobre las personas a las que estás ayudando,
y deja en claro que hay muchas maneras de involucrarse.
(Discutiremos esto con más detalle en la siguiente sección).

Tu mejor fuente de nuevos reclutas son tus propias clases:
``ver uno, hacer uno, enseñar uno'' ha funcionado bien para organizaciones voluntarias
durante el tiempo que ha \emph{habido} organizaciones voluntarias.
Asegúrate que cada clase u otro encuentro
termina diciendole a las personas cómo pueden ayudar y que su ayuda será bienvenida.
Las personas que vienen a ti de esta manera sabrán lo que haces
y tienen la experiencia reciente de ser receptores de lo que ofreces,
lo que ayuda a tu organización a evitar el punto ciego experto colectivo (\chapref{s:memory}).

\begin{aside}{Empieza pequeño}
  \hreffoot{https://en.wikipedia.org/wiki/Ben\_Franklin\_effect}{Ben Franklin} observó que
  una persona que le ha hecho un favor a alguien
  es más probable que les vuelva a hacer otro favor
  que alguien que había recibido un favor de esa persona.
  Por lo tanto, pedirle a la gente que haga algo pequeño por ti
  es un buen paso para lograr que hagan algo más grande.
  Una forma natural de hacer esto al enseñar
  es pedirle a la gente que corrijan la redacción o errores ortográficos en los materiales de tus lecciones,
  o que sugieran nuevos ejercicios o ejemplos.
  Si tus materiales están escritos de una manera mantenible (\secref{s:process-maintainability}),
  les da la oportunidad de practicar algunas habilidades útiles
  y te da la oportunidad de comenzar una conversación
  que podría conducir a una nueva incorporación a tu organización.
\end{aside}


La mitad del ciclo de vida voluntario es la incorporación y la retención,
que cubriremos en las Secciones~\ref{s:community-onboarding} y~\ref{s:community-retención}.
El último paso es cuando un miembro deja de ser parte de la organización:\index{retirement (of members)}
eventualmente, todas las personas se mueven,
y las organizaciones saludables planean ese momento.
Algunas cosas simples pueden hacer que, tanto la persona que se va, como todas las que se quedan
se sientan de forma positiva sobre el cambio:

\begin{description}

\item[Pide a las personas que sean explícitas sobre su partida.]
  para que todos sepan que realmente se han ido.

\item[Asegúrate de que no se sientan con verguenza de irse]
  o sobre cualquier otra cosa.

\item[Dales la oportunidad de transmitir sus conocimientos.]
  Por ejemplo,
  puedes pedirles que mentoreen a alguien durante algunas semanas como su última contribución,
  o que alguien se queda en la organización les realice una entrevista 
  para recopilar cualquier historia que valga la pena volver a contar.

\item[Asegúrate que entreguen las llaves.]
  Es incómodo descubrir seis meses después de que alguien se fué
  que es la única persona que sabe cómo reservar un lugar para el picnic anual.

\item[Contáctalos 2 a 3 meses después de que se vayan]
  para ver si tienen más ideas sobre lo que funcionó y lo que no funcionó mientras estuvieron contigo,
  o algún consejo para ofrecer que tampoco pensaron dar
  o que se sentían incómodos de dar mientras salian por la puerta.

\item[Agradeceles,]
  tanto cuando se van como la próxima vez que tu grupo se reúna.

\end{description}

\begin{aside}{Un manual que falta}
  Se han escrito miles de libros sobre cómo iniciar una empresa.
  Solo unos pocos describen cómo terminar una o como dejarla con gracia,
  a pesar de que hay un final para cada comienzo.
  Si alguna vez escribes uno,
  por favor hacemelo saber.
\end{aside}

\seclbl{Incorporación}{s:community-onboarding}

Después de decidir formar parte de un grupo,
la gente necesita ponerse al día,
y \cite{Shol2019} resume lo que sabemos sobre hacer esto.\index{onboarding (of members)}
La primera regla es tener y hacer cumplir un código de conducta (\secref{s:classroom-coc}),\index{Código de Conducta}
y encontrar una parte independiente que esté dispuesta a recibir y revisar informes de comportamiento inapropiado.
Alguien fuera de la organización tendrá la objetividad que los miembros de la organización pueden carecer,
y puede proteger a quienes reporten los incidentes para que no duden en plantear problemas relacionados con las personas 
encargadas del proyecto por temor a represalias o daños a su reputación.
El equipo que lidera el proyecto debe publicar las decisiones donde se aplique el código de conducta
para que la comunidad reconozca que el código es significativo.

La siguiente regla más importante es ser amigable.
Como dijo Fogel ~\cite{Foge2005},
``Si un proyecto no ahce una primera buena impresión,
los recién llegados van a esperar mucho tiempo antes de darle una segunda oportunidad.''
Otros autores han confirmado empíricamente la importancia de los entornos sociales amables y educados
en proyectos abiertos~\cite{Sing2012,Stei2013,Stei2018}:

\begin{description}

\item[Publica un mensaje de bienvenida]
  en las páginas de redes sociales, canales de Slack, foros o listas de correo electrónico del proyecto.
  Los proyectos podrían considerar mantener un canal o lista de ``Bienvenida'' exclusivos para ese fin,
  donde alguna de las personas que lidera el proyecto o gestiona la comunidad escribe una breve publicación pidiendo a los recién llegados que se presenten.

\item[Ayuda a las personas a encontrar una manera de hacer una contribución inicial,]
  como etiquetar lecciones particulares o talleres que necesitan trabajo como ``adecuados para los recién llegados''
  y pidiendo a los miembros ya establecidos que no los arreglen
  para asegurar que haya lugares adecuados para que los recién llegados comiencen a trabajar.

\item[Dirije a las personas recién llegadas a otras personas del proyecto similares a ellas]
  para demostrarles que pertenecen.

\item[Indícale los recursos esenciales del proyecto a las personas recién llegadas]
  como las pautas de contribución.

\item[Designa una o dos personas del proyecto como contacto.]
  para cada nueva presona recien llegada.
  Hacer esto puede hacer que quienes recién llegan sean menos reacios a hacer preguntas.

\end{description}

Una tercera regla que ayuda a todas las personas (no solo a quienes recien llegan)
es hacer que el conocimiento se pueda encontrar y mantenerlo actualizado.
Las personas nuevas son como exploradores que deben orientarse dentro de un paisaje desconocido~\cite{Dage2010}.
La información que se distribuye generalmente hace que las nuevas personas se sientan perdidas y desorientadas.
Dadas las diferentes posibilidades de lugares para mantener la información
(por ejemplo, \wikis, archivos en control de versiones, documentos compartidos, tweets antiguos o mensajes de Slack y archivos de correo electrónico)
es importante mantener la información sobre un tema específico consolidada en un solo lugar
para que las personas nuevas no necesiten navegar por múltiples fuentes de datos para encontrar lo que necesitan.
Organizar la información hace que los personas recién llegadas tengan más confianza y mejor orientación~\cite{Stei2016}.

Finalmente,
reconoce las primeras contribuciones de quienes recién inician
y piensa dónde y cómo podrían ayudar a largo plazo.
Una vez que han realizado exitosamente su primera contribución,
es probable que ambos tengan una mejor idea de lo que tienen para ofrecer
y cómo el proyecto puede ayudarlos.
Ayuda a las personas nuevas a encontrar el siguiente problema en el que tal vez quieran trabajar
o guíalos al siguiente tema que podrían disfrutar leyendo.
En particular,
animarles a ayudar a la próxima ola de nuevas personas
es una buena manera de reconocer lo que han aprendido
y una forma efectiva de transmitirlo.

\seclbl{Retención}{s:community-retention}
\index{retención (de participantes)}

\begin{quote}

 Si tu gente no se divierte, algo está muy mal.\\
  --- Saul Alinsky\index{Alinsky, Saul}

\end{quote}

Quienes participan de la comunidad no deberían esperar disfrutar cada momento de su trabajo con tu organización,
pero si no disfrutan nada de eso,
no se quedarán.
El disfrute no necesariamente significa tener una fiesta anual:
la gente puede disfrutar cocinar,
entrenar a otras personas,
o simplemente trabajar en silencio junto a otros y otras.
Hay varias cosas que toda organización debe hacer para garantizar
que las personas obtienen algo que valoran de su trabajo:

\begin{description}

\item[Pregunta a las personas qué quieren en vez de adivinar.]
  Asi como no sos tus estudiantes(\secref{s:process-personas}),
  probablemente seas diferente de otras personas de tu organización.
  Pregúntales a las personas qué quieren hacer,
  que se sienten cómodas haciendo (que puede no ser lo mismo),
  y qué limitaciones de tiempo tienen.
  Pueden llegar a decir: ``Cualquier cosa''.
  pero incluso una breve conversación probablemente ayude a descubrir el hecho de que
  les gusta interactuar con las personas, pero prefiere no administrar las finanzas del grupo
  o viceversa.

\item[Proporcionar muchas formas de contribuir.]
    Cuantas más formas haya para que las personas ayuden, más personas podrán hacerlo.
  Alguien a quien no le gusta estar frente a una audiencia 
  puede mantener el sitio web de su organización, 
  manejar sus cuentas 
  o corregir las lecciones.

\item[Reconoce las contribuciones.]
  A todos y todas nos gusta que nos aprecien, 
  así que las comunidades deben reconocer las contribuciones 
  de sus miembros, tanto en público como en privado, 
  mencionándolas en presentaciones, 
  poniéndolas en el sitio web, etc.
  Cada hora que alguien le haya dado a tu proyecto 
  puede ser una hora quitada de su vida personal o de su empleo oficial; 
  reconoce ese hecho 
  y deja en claro que, si bien más horas serían bienvenidas, 
  no esperas que hagan sacrificios insostenibles.

\item[Haz espacio.]
  Crees que estás siendo útil, 
  pero intervenir en cada decisión priva a las personas de su autonomía, 
  lo que reduce su motivación (\secref{s:motivation}).
  En particular, si siempre eres quien reponde primero a correos electrónicos o mensajes de chat, 
  las personas tienen menos oportunidades de crecer como miembros 
  y crear colaboraciones horizontales.
  Como resultado, 
  la comunidad continuará centrada en una o dos personas 
  en lugar de convertirse en una red altamente conectada 
  en la que otros y otras se sientan cómodos participando.

\end{description}

Otra forma de recompensar la participación es ofrecer capacitación.
Las organizaciones necesitan presupuestos, propuestas de subvenciones y resolución de disputas.
A la mayoría de las personas nunca se les enseña cómo hacer estas tareas más de lo que se les enseña a enseñar,
así que la oportunidad de adquirir habilidades transferibles 
es una razón poderosa para que las personas se involucren y se mantengan involucradas.
Si vas a hacer esto, no intentes proporcionar la capacitación tú mismo
a menos que sea en lo que te especialices.
Muchos grupos cívicos y comunitarios tienen programas de este tipo
y probablemente puedas llegar a un acuerdo con alguno de ellos.


Finally, 
while volunteers can do a lot,
tasks like system administration and accounting eventually need paid staff.
When you reach this point,
either pay people nothing or pay them a proper wage.
If you pay them nothing,
their real reward is the satisfaction of doing good;
if you pay them a token amount,
on the other hand,
you take that away without giving them the satisfaction of earning a living.

Finalmente,
mientras que las personas voluntarias pueden hacer mucho, 
tareas como la administración del sistema y la contabilidad eventualmente necesitan personal remunerado.
Cuando llegue a este punto, no pagues nada paga un salario adecuado.
Si no les paga nada, su verdadera recompensa es la satisfacción de hacer el bien;
por otro lado, si les pagas una cantidad simbólica, le quitas esa satisfacción sin darles la posibilidad de ganarse la vida.

\seclbl{Governanza}{s:community-governance}
\index{governance}

Cada organización tiene una estructura de poder:
la única pregunta es si es formal y explicable o informal y, por lo tanto, inexplicable~\cite{Free1972}.
Esta última forma, en realidad, funciona bastante bien para grupos de hasta media docena de personas 
en las que todos y todas se conocen.
Más allá de esa cantidad, 
necesitas reglas para explicar 
quién tiene la autoridad para tomar qué decisiones 
y cómo lograr consenso (\secref{s:meetings-marthas-rules}).

El modelo de gobierno que prefiero se denomina \gref{g:commons}{commons} (como en un bien común),
donde la administración se realiza conjuntamente por la comunidad, de acuerdo con las reglas que ella misma ha desarrollado y adoptado~\cite{Ostr2015}.
Como subraya ~\cite{Boll2014}, las tres partes de esa definición son esenciales:
commons no es solo una pastura compartida o un conjunto de bibliotecas de software, 
sino que también incluye a la comunidad que lo comparte y las reglas que usan para hacerlo.

Los modelos más populares son las corporaciones con fines de lucro y las organizaciones sin fines de lucro; 
la mecánica varía de una jurisdicción a otra, 
por lo que debes buscar asesoramiento antes de elegir\footnote{
  Este es uno de los momentos 
  en que vale la pena tener vínculos con el gobierno local u otras organizaciones afines.}.
Ambos tipos de organización tienen la máxima autoridad en su junta o directorio.
En términos generales, se trata de un \gref{g:service-board}{directorio o junta de servicio} 
cuyos miembros también asumen otras funciones en la organización 
o un \gref{g:governance-board}{directorio de gobernanza} cuya responsabilidad principal es contratar, supervisar 
y, si es necesario, despedir al director.
Los miembros de la junta pueden ser elegidos por la comunidad o nombrados; 
en cualquier caso, es importante priorizar la capacidad sobre la pasión 
(la última es más importante para la base de la organización) 
y tratar de reclutar habilidades particulares como contabilidad, marketing, etc.

\begin{aside}{Elige la democracia}
  Cuando llegue el momento, 
  haz de tu organización una democracia:
  tarde o temprano (generalmente más temprano que tarde), 
  cada junta designada se convierte en una sociedad de mutuo acuerdo.
  Darle poder a sus miembros es complicado, 
  pero es la única forma inventada hasta ahora para garantizar 
  que las organizaciones continúen satisfaciendo las necesidades reales de las personas.
\end{aside}

\seclbl{Cuídate}{s:community-care}

El Síndrome de desgaste ocupacional (bournout en inglés) es un riesgo crónico en cualquier actividad comunitaria~\cite{Pign2016},\index{burnout}
así que aprende a decir no más seguido de lo que dices sí.
Si no te cuidas,
no podrás cuidar a tu comunidad.

\begin{aside}{Quedándose sin ``No''}
  Investigaciones en la década de 1990 parecían mostrar que nuestra capacidad de ejercer fuerza de voluntad es finita:
  si tenemos que resistirnos a comer la última dona en la caja cuando tenemos hambre,
  somos menos propensos a doblar la ropa y viceversa.
  Este fenómeno se llama \gref{g:ego-depletion}{agotamiento del ego},
  y si bien los estudios recientes no han podido replicar esos primeros resultados~\cite{Hagg2016},
  decir ``sí'' cuando estamos demasiado cansados ​​para decir ``no''
  es una trampa en la que caen muchos organizadores.
\end{aside}

Una forma de asegurarte de cumplir con tu ``no'' 
es escribir una lista de cosas que vale la pena hacer 
pero que \emph{no} vas a hacer.
Al momento de escribir este libro, mi lista incluye cuatro libros, 
dos proyectos de software,
el rediseño de mi sitio web personal,
y aprender a tocar el silbato.

Finalmente,
recuerda de vez en cuando que
eventualmente toda organización necesita ideas y liderazgo nuevos.
Cuando llegue ese momento,
entrena a tus sucesores y continúa con la mayor gracia posible.
Indudablemente harán cosas que tú no harías,
pero pocas cosas en la vida son tan satisfactorias como
ver algo que ayudaste a construir adquiere vida propia.
Celebra eso --- no tendrás ningún problema para encontrar otra cosa que te mantenga ocupado/a.

\seclbl{Ejercicios}{s:community-exercises}

Varios de estos ejercicios se toman de~\cite{Brow2007}.

\exercise{¿Qué tipo de comunidad?}{individual}{15}

Vuelve a leer la descripción de los cuatro tipos de comunidades.
y decide cuál/cuales es o aspira ser su grupo.

\exercise{Personas que puedes conocer}{grupos pequeños}{30}

Como organizador/a,
a veces, parte de tu trabajo, es ayudar a las personas a encontrar una manera de contribuir a pesar de sí mismas.
En pequeños grupos,
elige tres de las personas a continuación
y discute cómo los ayudarías a convertirse en una mejor contribuyente para tu organización.

\begin{description}

\item[Anna]
  sabe más sobre cada tema que todas las demás personas juntas --- al menos,
  ella cree que lo hace.
  No importa lo que digas,
  ella te corregirá;
  no importa lo que sepas, ella lo sabe mejor.
	
\item[Catherine]
  tiene tan poca confianza en su propia habilidad
  que no tomará ninguna decisión,
  sin importar que tan pequeña sea,
  hasta que no haya consultado con alguien más.

\item[Frank]
  disfruta saber cosas que otras personas no saben.
  Puede hacer milagros,
  pero cuando se le pregunta cómo lo hizo,
  sonreirá y dirá:
  ``Oh, estoy seguro de que puedes resolverlo''.

\item[Hediyeh]
  es tranquila.
  Nunca habla en las reuniones,
  incluso cuando sabe que otras personas están equivocadas.
  Podría contribuir a la lista de correo,
  pero es muy sensible a las críticas
  y siempre retrocede en lugar de defender su punto.

\item[Ken]
  aprovecha el hecho de que la mayoría de las personas preferiría hacer su parte del trabajo
  que quejarse de él.
  Lo frustrante es que es tan plausible cuando alguien finalmente lo confronta.
  `Ha habido errores en todos lados,''
  dice él,
  o, ``Bueno, creo que estás siendo un poco quisquilloso.''.

\item[Melissa]
  tiene buenas intenciones
  pero de alguna manera siempre surge algo
  y sus tareas nunca terminan hasta el último momento posible.
  Por supuesto,
  eso significa que todos los que dependen de ella no pueden hacer su trabajo
  hasta \emph{después} del último momento posible {\ldots}
  
\item[Raj]
  Es grosero.
  ``Así es la forma en que hablo'', dice.
  ``Si no puedes hackearlo, ve a buscar otro equipo''.
  Su frase favorita es: ``Eso es estúpido''.
  y usa una obscenidad en cada segunda oración.

\end{description}

\exercise{Valores}{grupos pequeños}{45}

Responde estas preguntas por tu cuenta y
luego compara tus respuestas con las de los demás.

\begin{enumerate}

\item
  ¿Cuáles son los valores que expresa tu organización?

\item
  ¿Son estos los valores que deseas que la organización exprese?

\item
  Si tu respuesta es no, ¿qué valores te gustaría expresar?

\item
  ¿Cuáles son los comportamientos específicos que demuestran esos valores?

\item
  ¿Qué comportamientos demostrarían lo contrario de esos valores?
\end{enumerate}

\exercise{Procedimientos de reuniones}{grupos pequeños}{30}

Responde estas preguntas por tu cuenta y
luego compara tus respuestas con las de los demás.

\begin{enumerate}

\item
  ¿Cómo se llevan a cabo las reuniones?

\item
  ¿Es así como quieres que se realicen tus reuniones?

\item
  ¿Las reglas para ejecutar reuniones son explícitas o simplemente se asumen?

\item
  ¿Estas son las reglas que quieres?

\item
  ¿Quién es elegible para votar o tomar decisiones?

\item
  ¿Son estas personas las que quieres que se le otorgue autoridad para tomar decisiones?

\item
  ¿Utilizan la regla de la mayoría, toman decisiones por consenso u otra cosa?

\item
  ¿Es así como quieres tomar decisiones?

\item
  ¿Cómo saben las personas en una reunión cuándo se ha tomado una decisión?

\item
  ¿Cómo saben las personas que no estuvieron en una reunión qué decisiones se tomaron?

\item
  ¿Funciona esto para tu grupo?

\end{enumerate}

\exercise{Tamaño}{grupos pequeños}{20}

Responde estas preguntas por tu cuenta y
luego compara tus respuestas con las de los demás.


\begin{enumerate}

\item
¿Qué tan grande es tu grupo?

\item
  ¿Es este el tamaño que deseas para su organización?

\item
  Si tu repsuesta es no, ¿de qué tamaño te gustaría que fuera?

\item
  ¿Tienes algún límite en cuanto a la cantidad de miembros?

\item
  ¿Te beneficiarías de establecer ese límite?

\end{enumerate}

\exercise{Convertirse en miembro}{grupos pequeños}{45}

Responde estas preguntas por tu cuenta y
luego compara tus respuestas con las de los demás.


\begin{enumerate}

\item
  ¿Cómo se une alguien a tu gupo?

\item
  ¿Qué tan bien funciona este proceso?

\item
	¿Hay cuotas de membresía?

\item
  ¿Se requiere que las personas estén de acuerdo con alguna regla de comportamiento al unirse?

\item
  ¿Son estas las reglas de comportamiento que quieres?

\item
  ¿Cómo descubren las personas recien llegadas lo que hay que hacer?

\item
  ¿Qué tan bien funciona este proceso?
  
\end{enumerate}

\exercise{Dotación de personal}{grupos pequeños}{30}

Responde estas preguntas por tu cuenta y
luego compara tus respuestas con las de los demás.


\begin{enumerate}

\item
  ¿Tiene personal pagado en su organización?
  o son todos y todas voluntarios?

\item
  ¿Deberías tener personal pago?

\item
  ¿Quieres / necesitas más o menos personal?

\item
  ¿Qué hacen los miembros del personal?

\item
  ¿Son estos los roles y funciones principales que necesitas que el personal desempeñe?

\item
  ¿Quién supervisa a tu personal?

\item
  ¿Es este el proceso de supervisión que quieres para tu grupo?

\item
  ¿Cuánto le pagan a tu personal?

\item
  ¿Es este el salario adecuado para realizar el trabajo necesario?

\end{enumerate}

\exercise{Dinero}{grupos pequeños}{30}

Responde estas preguntas por tu cuenta y
luego compara tus respuestas con las de los demás.

\begin{enumerate}

\item
  ¿Quién paga y por qué?

\item
  Is this who you want to be paying?
  ¿Es a él a quien le quieres pagar?
  ¿Es por esto y a quien le quieres pagar?

\item
  ¿De dónde sacas/obtienes tu dinero?

\item
  ¿Es así como quieres obtener tu dinero?

\item
  Si no, ¿tiene algún plan para hacerlo de otra manera?

\item
  Si es así, ¿cuáles son esos planes?

\item
  ¿Quién está siguiendo estos planes para asegurarse de que suceda?

\item
  ¿Cuánto dinero tienes?

\item
  ¿Cuánto dinero necesitas?

\item
  ¿En qué gastas la mayor parte de tu dinero?

\item
  ¿Es así como quieres gastar tu dinero?

\end{enumerate}

\exercise{Tomando ideas prestadas}{toda la clase}{15}

Muchas de mis ideas sobre cómo construir una comunidad 
han sido moldeadas por mi experiencia en el desarrollo de software de código abierto.
\cite{Foge2005} (que es \hreffoot {http://producingoss.com/}{disponible en línea})
es una buena guía de lo que ha funcionado y lo que no ha funcionado para esas comunidades,
y el \hreffoot{https://opensource.guide/} {sitio de guías de código abierto}
tiene una gran cantidad de información útil también.
Elije una sección de este último recurso, como ``Encontrar usuarios para su proyecto (Finding Users for Your Project en inglés)''
o ``Liderazgo y gobernanza (Leadership and Governance en inglés)''
y presenta al grupo, en dos minutos, una idea 
que encontraste útil o con la que estuviste muy en desacuerdo.

\exercise{¿Quién eres tú?}{grupos pequeños}{20}

La Administración Nacional Oceánica y Atmosférica (NOAA por sus siglas en inglés)
tiene una guía breve, útil y divertida para
\hreffoot{https://coast.noaa.gov/ddb/}{lidiar con comportamientos disruptivos}.
Clasifica esos comportamientos bajo etiquetas como ``hablador'', ``indecisa'' y ``tímida''.
y describe estrategias para manejar cada una.
En grupos de 3 a 6 personas,
lean la guía y decidan cuál de estas descripciones les queda mejor.
¿Crees que las estrategias descritas para manejar personas como tú son efectivas?
¿Son otras estrategias igualmente o más efectivas?

\exercise{Creando lecciones entre todos y todas}{grupos pequeños}{30}

Una de las claves del éxito de \hreffoot{http://carpentries.org}{the Carpentries}
es su énfasis en construir y mantener lecciones en forma colaborativa~\cite{Wils2016,Deve2018}.
Trabajando en grupos de 3--4:

\begin{enumerate}

\item
  Elije una lección breve que todos/as hayan usado.

\item
  Haz una revisión cuidadosa para crear una única lista con sugerencias de mejoras.

\item
  Ofrece esas sugerencias al autor de la lección.

\end{enumerate}

\exercise{¿Estás crujiente?}{individual}{10}

\hreffoot{https://mailchi.mp/d54702d0a790/take-my-horse-to-the-sand-hill-road}{Johnathan Nightingale escribió}:

\begin{quote}
  Cuando trabajaba en Mozilla, 
  utilizamos el término "crujiente" (crispy en inglés) para referirnos al estado justo antes de llegar al síndrome de desgaste ocupacional.
  Las personas que son crujientes no son divertidas.
  Son descorteses.
  Están ansiosos por una pelea que pueden ganar.
  Lloran sin mucha advertencia.
  {\ldots} reconoceríamos lo "crujiente" de nuestros colegas y nos cuidaríamos mutuamente 
  [pero] es una cosa fea, que vimos tanto, que tuvimos un proceso cultural completo a su alrededor.
\end{quote}

\noindent
Responde ``sí'' o ``no'' a cada una de las siguientes preguntas.
¿Qué tan cerca estás de tener síndrome de desgaste ocupacional?

\begin{itemize}
\item ¿Te has vuelto cínica/o o crítica/o en el trabajo?
\item ¿Tienes que arrastrarte al trabajo o tienes problemas para comenzar a trabajar?
\item ¿Te has vuelto irritable o impaciente con tus compañeros de trabajo?
\item ¿Te resulta difícil concentrarte?
\item ¿No logras satisfacción de tus logros?
\item ¿Estás usando comida, drogas o alcohol para sentirte mejor o simplemente no sentir?
\end{itemize}
