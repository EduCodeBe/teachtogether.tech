\chapter{Construyendo una comunidad de práctica}\label{s:community}

No tienes que arreglar todos los males de la sociedad para enseñar programación,
pero \emph{si} te tienes que involucrar en
lo que sucede fuera de tu clase si quieres que las personas aprendan.
Esto se aplica tanto a las personas que enseñan como a las que aprenden:
muchos docentes free-range comienzan como voluntarios o como trabajadores a medio tiempo
y deben hacer malabares con sus clases y muchos otros compromisos.
Lo que sucede fuera del aula es tan importante para su éxito 
como lo es para sus estudiantes,
así que la mejor manera de ayudar a ambos es fomentar una comunidad de enseñanza.


\begin{aside}{Finlandia y por qué no}
Las escuelas de Finlandia se encuentran entre las más exitosas del mundo,
  pero como Anu Partanen\index{Partanen, Anu}
  \hreffoot{https://www.theatlantic.com/national/archive/2011/12/what-americans-keep-ignoring-about-finlands-school-success/250564/}{señaló},
  no han tenido éxito solas.
  Los intentos de otros países por adoptar métodos de enseñanza finlandeses están condenados al fracaso
  a menos que esos países también garanticen que los niños (y sus padres) estén seguros,
  bien alimentados,
  y tratados justamente por las cortes~\cite{Sahl2015, Wilk2011}.
  Esto no es ninguna sorpresa dado lo que sabemos sobre la importancia de la motivación para el aprendizaje (\chapref{s:motivation}):
  todas las personas lo harán peor si creen que el sistema es impredecible, injusto o indiferente.
\end{aside}

Un marco para pensar sobre las comunidades de enseñanza es el \gref{g:situado-learning}{aprendizaje situado},
que se centra en cómo una \gref{g:legitimate-peripheral-participation}{participación periférica legítima}
lleva a las personas a convertirse en miembros de
una \gref{g:community-of-practice}{comunidad de práctica}~\cite{Weng2015}.
Analizando esos términos,
una comunidad de práctica es un grupo de personas unidas por su interés en alguna actividad,
como tejer o la física de partículas.
La participación periférica legítima significa realizar tareas simples y de bajo riesgo
que la comunidad reconoce como contribuciones válidas:
hacer tu primera bufanda,
llenar sobres durante una campaña electoral,
o revisar documentación de software de código abierto.

Situated learning focuses on the transition from being a newcomer
to being accepted as a peer by those who are already community members.
This typically means starting with simple tasks and tools,
then doing similar tasks with more complex tools,
and finally tackling the same work as advanced practitioners.
For example,
children learning music may start by playing nursery rhymes on a recorder or ukulele,
then play other simple songs on a trumpet or saxophone in a band,
and finally start exploring their own musical tastes.
Common ways to support this progression include:

El aprendizaje situado se centra en la transición de ser un recién llegado
a ser aceptado como un compañero por aquellos que ya son miembros de la comunidad.
Esto generalmente significa comenzar con tareas y herramientas simples,
luego hacer tareas similares con herramientas más complejas,
y finalmente abordar el mismo trabajo que los practicantes avanzados.
Por ejemplo,
los niños que aprenden música pueden comenzar tocando canciones infantiles en una grabadora o un ukelele,
luego tocan otras canciones simples en una trompeta o saxofón en una banda,
y finalmente comienzan a explorar sus propios gustos musicales.
Las formas comunes de apoyar esta progresión incluyen:

\newpage
\begin{description}

\item[Resolución de problemas:]
  ``No puedo avanzar --- ¿podemos trabajar en el diseño de esta lección juntos? ''
\item[Requests for information:]
  ``¿Cuál es la contraseña para el administrador de la lista de correo? ''
\item[Búsqueda de experiencia:]
  ``¿Alguien ha tenido un alumno con discapacidad para leer? ''
\item[Compartir recursos:]
  ``El año pasado armé un sitio web para una clase que puedes usar como punto de partida ''.
\item[Coordinación:]
  ``¿Podemos hacer nuestros pedidos de camisetas juntos para obtener un descuento? ''
\item[Construir un argumento:]
  ``Será más fácil convencer a mi jefe para que haga cambios si sé cómo otros campistas hacen esto ''.
\item[Documentar proyectos:]
  ``Hemos tenido este problema cinco veces ahora. Vamos a escribirlo de una vez por todas ''.
\item[Mapeo del conocimiento:]
  ``¿Qué otros grupos están haciendo cosas como esta en vecindarios o ciudades cercanas? ''
\item[Visitas:]
  ``¿Podemos venir a ver su programa extracurricular? Necesitamos establecer uno en nuestra ciudad ''.
\end{description}

\hreffoot{https://www.feverbee.com/types-of-community-and-activity-within-the-community/}{En términos generales},
Una comunidad de práctica puede ser:

\begin{description}
\item[Comunidad de acción:]
  personas enfocadas en un objetivo compartido,
  como conseguir que un candidato sea elegido.

\item[Comunidad de interés:]
  los miembros se unen por un problema compartido,
  como tratar una enfermedad a largo plazo.

\item[Comunidad de interés:]
  enfocado en un amor compartido por algo como el backgammon o tejer.

\item[Comunidad de lugar:]
  personas que viven o trabajan juntas.

\end{description}
  
La mayoría de las comunidades son mezclas de estos tipos,
como las personas en Toronto a las que les gusta enseñar tecnología.
El enfoque de una comunidad también puede cambiar con el tiempo:
por ejemplo,
un grupo de apoyo para personas que padecen depresión (comunidad de interés)
puede decidir recaudar fondos para mantener una línea de ayuda (comunidad de acción).
Mantener el funcionamiento la línea de ayuda puede convertirse en el foco del grupo (comunidad de interés).

\begin{aside}{Sopa, luego himnos}
  Los manifiestos son divertidos de escribir,
  pero la mayoría de las personas se unen a una comunidad de voluntarios para ayudar y ser ayudados
  en lugar de discutir sobre la redacción de una declaración de visión\footnote{
    Las personas que prefieren lo último a menudo están \emph{solo} interesadas en discutir{\ldots}}.
  Por lo tanto, debes centrarte en
  qué personas pueden crear lo que otros miembros de la comunidad usarán de inmediato.
  Una vez que tu organización demuestre que puede lograr cosas pequeñas,
  la gente estará más segura de que vale la pena ayudarte con proyectos más grandes.
  Es el momento de preocuparse por definir los valores que guiarán a sus miembros.
\end{aside}

\seclbl{Aprende, luego haz}{s:community-learn-then-do}

El primer paso para construir una comunidad es decidir si deberías construirla,
o si sería más efectivo unirte a una organización existente.
Miles de grupos ya están enseñando habilidades tecnológicas a las personas,
desde los \hreffoot{http://www.4-h-canada.ca/}{4-H Club}
y los \hreffoot{https://www.frontiercollege.ca/}{programas de alfabetización}
hasta organizaciones sin fines de lucro que te inician en la programación como
\hreffoot{http://www.blackgirlscode.com/}{Black Girls Code}
y \hreffoot{http://bridgeschool.io/}{Bridge}.
Unirse a un grupo existente te dará ventaja en la enseñanza,
un conjunto inmediato de colegas,
y una oportunidad de aprender más sobre cómo manejar las cosas;
Con suerte,
ir aprendiendo esas habilidades mientras haces una contribución inmediata
será más importante que poder decir que
sos el fundador o líder de algo nuevo.

Ya sea que te unas a un grupo existente o inicies uno propio,
serás más efectivo si lees un poco sobre como organizar una comunidad.
\cite{Alin1989,Lake2018} probablemente es el trabajo más conocido sobre organización de grupos de base,
mientras que ~\cite{Brow2007,Midw2010,Lake2018} son manuales prácticos basados ​​en décadas de experiencia.
Si quieres leer más profundamente,
\cite{Adam1975} es la historia de la Highlander Folk School,
cuyo enfoque ha sido emulado por muchos grupos exitosos,
mientras que ~\cite{Spal2014} es una guía para enseñar a adultos
escrita por alguien con profundas raíces personales en la organización
y \hreffoot{https://www.nonprofitready.org/}{NonprofitReady.org}
ofrece capacitación profesional gratuita.

\seclbl{Cuatro pasos}{s:community-four-steps}

Todas las personas que se involucren con tu organización
(incluyéndote)
pasa por cuatro fases:
reclutamiento, incorporación, retención y retiro.
No necesitas preocuparte por este ciclo cuando estés comenzando,
pero vale la pena pensar en este tema
tan pronto como más de un puñado de personas no fundadoras estén involucradas.

El primer paso es reclutar voluntarias y voluntarios.\index{recruitment (of members)}
Tu estrategia de marketing debería ayudarte con esto haciendo que tu organización sea localizable\index{findability!of organizations}
y que tu misión y valor sean claros
para las personas que quieran involucrarse

(\chapref{s:outreach}).
Comparte historias que ejemplifiquen el tipo de ayuda que desea
así como historias sobre las personas a las que estás ayudando,
y deja en claro que hay muchas maneras de involucrarse.
(Discutiremos esto con más detalle en la siguiente sección).

Tu mejor fuente de nuevos reclutas son tus propias clases:
``ver uno, hacer uno, enseñar uno'' ha funcionado bien para organizaciones voluntarias
durante el tiempo que ha \emph{habido} organizaciones voluntarias.
Asegúrate que cada clase u otro encuentro
termina diciendole a las personas cómo pueden ayudar y que su ayuda será bienvenida.
Las personas que vienen a ti de esta manera sabrán lo que haces
y tienen la experiencia reciente de ser receptores de lo que ofreces,
lo que ayuda a tu organización a evitar el punto ciego experto colectivo (\chapref{s:memory}).

\begin{aside}{Empieza pequeño}
  \hreffoot{https://en.wikipedia.org/wiki/Ben\_Franklin\_effect}{Ben Franklin} observó que
  una persona que le ha hecho un favor a alguien
  es más probable que les vuelva a hacer otro favor
  que alguien que había recibido un favor de esa persona.
  Por lo tanto, pedirle a la gente que haga algo pequeño por ti
  es un buen paso para lograr que hagan algo más grande.
  Una forma natural de hacer esto al enseñar
  es pedirle a la gente que corrijan la redacción o errores ortográficos en los materiales de tus lecciones,
  o que sugieran nuevos ejercicios o ejemplos.
  Si tus materiales están escritos de una manera mantenible (\secref{s:process-maintainability}),
  les da la oportunidad de practicar algunas habilidades útiles
  y te da la oportunidad de comenzar una conversación
  que podría conducir a una nueva incorporación a tu organización.
\end{aside}


La mitad del ciclo de vida voluntario es la incorporación y la retención,
que cubriremos en las Secciones~\ref{s:community-onboarding} y~\ref{s:community-retención}.
El último paso es cuando un miembro deja de ser parte de la organización:\index{retirement (of members)}
eventualmente, todas las personas se mueven,
y las organizaciones saludables planean ese momento.
Algunas cosas simples pueden hacer que, tanto la persona que se va, como todas las que se quedan
se sientan de forma positiva sobre el cambio:

\begin{description}

\item[Pide a las personas que sean explícitas sobre su partida.]
  para que todos sepan que realmente se han ido.

\item[Asegúrate de que no se sientan con verguenza de irse]
  o sobre cualquier otra cosa.

\item[Dales la oportunidad de transmitir sus conocimientos.]
  Por ejemplo,
  puedes pedirles que mentoreen a alguien durante algunas semanas como su última contribución,
  o que alguien se queda en la organización les realice una entrevista 
  para recopilar cualquier historia que valga la pena volver a contar.

\item[Asegúrate que entreguen las llaves.]
  Es incómodo descubrir seis meses después de que alguien se fué
  que es la única persona que sabe cómo reservar un lugar para el picnic anual.

\item[Contáctalos 2 a 3 meses después de que se vayan]
  para ver si tienen más ideas sobre lo que funcionó y lo que no funcionó mientras estuvieron contigo,
  o algún consejo para ofrecer que tampoco pensaron dar
  o que se sentían incómodos de dar mientras salian por la puerta.

\item[Agradeceles,]
  tanto cuando se van como la próxima vez que tu grupo se reúna.

\end{description}

\begin{aside}{Un manual que falta}
  Se han escrito miles de libros sobre cómo iniciar una empresa.
  Solo unos pocos describen cómo terminar una o como dejarla con gracia,
  a pesar de que hay un final para cada comienzo.
  Si alguna vez escribes uno,
  por favor hacemelo saber.
\end{aside}

\seclbl{Incorporación}{s:community-onboarding}

Después de decidir formar parte de un grupo,
la gente necesita ponerse al día,
y \cite{Shol2019} resume lo que sabemos sobre hacer esto.\index{onboarding (of members)}
La primera regla es tener y hacer cumplir un código de conducta (\secref{s:classroom-coc}),\index{Código de Conducta}
y encontrar una parte independiente que esté dispuesta a recibir y revisar informes de comportamiento inapropiado.
Alguien fuera de la organización tendrá la objetividad que los miembros de la organización pueden carecer,
y puede proteger a quienes reporten los incidentes para que no duden en plantear problemas relacionados con las personas 
encargadas del proyecto por temor a represalias o daños a su reputación.
El equipo que lidera el proyecto debe publicar las decisiones donde se aplique el código de conducta
para que la comunidad reconozca que el código es significativo.

La siguiente regla más importante es ser amigable.
Como dijo Fogel ~\cite{Foge2005},
``Si un proyecto no ahce una primera buena impresión,
los recién llegados van a esperar mucho tiempo antes de darle una segunda oportunidad.''
Otros autores han confirmado empíricamente la importancia de los entornos sociales amables y educados
en proyectos abiertos~\cite{Sing2012,Stei2013,Stei2018}:

\begin{description}

\item[Publica un mensaje de bienvenida]
  en las páginas de redes sociales, canales de Slack, foros o listas de correo electrónico del proyecto.
  Los proyectos podrían considerar mantener un canal o lista de ``Bienvenida'' exclusivos para ese fin,
  donde alguna de las personas que lidera el proyecto o gestiona la comunidad escribe una breve publicación pidiendo a los recién llegados que se presenten.

\item[Ayuda a las personas a encontrar una manera de hacer una contribución inicial,]
  como etiquetar lecciones particulares o talleres que necesitan trabajo como ``adecuados para los recién llegados''
  y pidiendo a los miembros ya establecidos que no los arreglen
  para asegurar que haya lugares adecuados para que los recién llegados comiencen a trabajar.

\item[Dirije a las personas recién llegadas a otras personas del proyecto similares a ellas]
  para demostrarles que pertenecen.

\item[Indícale los recursos esenciales del proyecto a las personas recién llegadas]
  como las pautas de contribución.

\item[Designa una o dos personas del proyecto como contacto.]
  para cada nueva presona recien llegada.
  Hacer esto puede hacer que quienes recién llegan sean menos reacios a hacer preguntas.

\end{description}

Una tercera regla que ayuda a todas las personas (no solo a quienes recien llegan)
es hacer que el conocimiento se pueda encontrar y mantenerlo actualizado.
Las personas nuevas son como exploradores que deben orientarse dentro de un paisaje desconocido~\cite{Dage2010}.
La información que se distribuye generalmente hace que las nuevas personas se sientan perdidas y desorientadas.
Dadas las diferentes posibilidades de lugares para mantener la información
(por ejemplo, \wikis, archivos en control de versiones, documentos compartidos, tweets antiguos o mensajes de Slack y archivos de correo electrónico)
es importante mantener la información sobre un tema específico consolidada en un solo lugar
para que las personas nuevas no necesiten navegar por múltiples fuentes de datos para encontrar lo que necesitan.
Organizar la información hace que los personas recién llegadas tengan más confianza y mejor orientación~\cite{Stei2016}.

Finalmente,
reconoce las primeras contribuciones de quienes recién inician
y piensa dónde y cómo podrían ayudar a largo plazo.
Una vez que han realizado exitosamente su primera contribución,
es probable que ambos tengan una mejor idea de lo que tienen para ofrecer
y cómo el proyecto puede ayudarlos.
Ayuda a las personas nuevas a encontrar el siguiente problema en el que tal vez quieran trabajar
o guíalos al siguiente tema que podrían disfrutar leyendo.
En particular,
animarles a ayudar a la próxima ola de nuevas personas
es una buena manera de reconocer lo que han aprendido
y una forma efectiva de transmitirlo.

\seclbl{Retención}{s:community-retention}
\index{retención (de participantes)}

\begin{quote}

 Si tu gente no se divierte, algo está muy mal.\\
  --- Saul Alinsky\index{Alinsky, Saul}

\end{quote}

Quienes participan de la comunidad no deberían esperar disfrutar cada momento de su trabajo con tu organización,
pero si no disfrutan nada de eso,
no se quedarán.
El disfrute no necesariamente significa tener una fiesta anual:
la gente puede disfrutar cocinar,
entrenar a otras personas,
o simplemente trabajar en silencio junto a otros y otras.
Hay varias cosas que toda organización debe hacer para garantizar
que las personas obtienen algo que valoran de su trabajo:

\begin{description}

\item[Pregunta a las personas qué quieren en vez de adivinar.]
  Asi como no sos tus estudiantes(\secref{s:process-personas}),
  probablemente seas diferente de otras personas de tu organización.
  Pregúntales a las personas qué quieren hacer,
  que se sienten cómodas haciendo (que puede no ser lo mismo),
  y qué limitaciones de tiempo tienen.
  Pueden llegar a decir: ``Cualquier cosa''.
  pero incluso una breve conversación probablemente ayude a descubrir el hecho de que
  les gusta interactuar con las personas, pero prefiere no administrar las finanzas del grupo
  o viceversa.

\item[Proporcionar muchas formas de contribuir.]
  The more ways there are for people to help,
  the more people will be able to.
  Someone who doesn't like standing in front of an audience
  may be able to maintain your organization's website,
  handle its accounts,
  or proofread lessons.
  
  Cuantas más formas haya para que las personas ayuden, más personas podrán hacerlo.
  Alguien a quien no le gusta estar frente a una audiencia puede mantener el sitio web de su organización, manejar sus cuentas o corregir las lecciones.

\item[Reconoce las contribuciones.]
  Everyone likes to be appreciated,
  so communities should acknowledge
  their members' contributions both publicly and privately
  by mentioning them in presentations,
  putting them on the website,
  and so on.
  Every hour that someone has given your project
  may be an hour taken away from their personal life or their official employment;
  recognize that fact
  and make it clear that while more hours would be welcome,
  you do not expect them to make unsustainable sacrifices.

\item[Make space.]
  You think you're being helpful,
  but intervening in every decision robs people of their autonomy,
  which in return reduces their motivation (\secref{s:motivation}).
  In particular,
  if you're always the first one to reply to email or chat messages,
  people have less opportunity to grow as members
  and to create horizontal collaborations.
  As a result,
  the community will continue to be centered around one or two individuals
  rather than becoming a highly connected network
  in which others feel comfortable participating.

\end{description}

Another way to reward participation is to offer training.
Organizations need budgets, grant proposals, and dispute resolution.
Most people are never taught how to do this any more than they are taught how to teach,
so the opportunity to gain transferable skills
is a powerful reason for people to get and stay involved.
If you are going to do this,
don't try to provide the training yourself
unless it's what you specialize in.
Many civic and community groups have programs of this kind,
and you can probably make a deal with one of them.

Finally, 
while volunteers can do a lot,
tasks like system administration and accounting eventually need paid staff.
When you reach this point,
either pay people nothing or pay them a proper wage.
If you pay them nothing,
their real reward is the satisfaction of doing good;
if you pay them a token amount,
on the other hand,
you take that away without giving them the satisfaction of earning a living.

\seclbl{Governance}{s:community-governance}
\index{governance}

Every organization has a power structure:
the only question is
whether it's formal and accountable or informal and therefore unaccountable~\cite{Free1972}.
The latter actually works pretty well for groups of up to half a dozen people
in which everyone knows everyone else.
Beyond that,
you need rules to spell out
who has the authority to make which decisions
and how to achieve consensus (\secref{s:meetings-marthas-rules}).

The governance model I prefer is a \gref{g:commons}{commons},
which is something managed jointly by a community
according to rules they themselves have evolved and adopted~\cite{Ostr2015}.
As~\cite{Boll2014} emphasizes,
all three parts of that definition are essential:
a commons isn't just a shared pasture or a set of software libraries,
but also includes the community that shares it
and the rules they use to do so.

For-profit corporations and incorporated nonprofits are more popular models;
the mechanics vary from jurisdiction to jurisdiction,
so you should seek advice before choosing\footnote{
  This is one of the times when
  having ties with local government or other like-minded organizations pays off.}.
Both kinds of organization vest ultimate authority in their board.
Broadly speaking, this is either a \gref{g:service-board}{service board}
whose members also take on other roles in the organization
or a \gref{g:governance-board}{governance board} whose primary responsibility is to hire, monitor,
and if need be fire the director.
Board members can be elected by the community or appointed;
in either case,
it's important to prioritize competence over passion
(the latter being more important for the rank and file)
and to try to recruit for particular skills such as accounting, marketing, and so on.

\begin{aside}{Choose Democracy}
  When the time comes,
  make your organization a democracy:
  sooner or later (usually sooner),
  every appointed board turns into a mutual agreement society.
  Giving your members power is messy,
  but is the only way invented so far to ensure that
  organizations continue to meet people's actual needs.
\end{aside}

\seclbl{Look After Yourself}{s:community-care}

Burnout is a chronic risk in any community activity~\cite{Pign2016},\index{burnout}
so learn to say no more often than you say yes.
If you don't take care of yourself,
you won't be able to take care of your community.

\begin{aside}{Running Out of ``No''}
  Research in the 1990s seemed to show that our ability to exert willpower is finite:
  if we have to resist eating the last donut in the box when we're hungry,
  we are less likely to fold laundry and vice versa.
  This phenomenon is called \gref{g:ego-depletion}{ego depletion},
  and while recent studies have failed to replicate those early results~\cite{Hagg2016},
  saying ``yes'' when we're too tired to say ``no''
  is a trap many organizers fall into.
\end{aside}

One way to make your ``no'' stick
is to write a to-don't list of things that would be worth doing
but which you \emph{aren't} going to do.
At the time of writing,
mine includes four books,
two software projects,
redesign of my personal website,
and learning to play the penny whistle.

Finally,
remind yourself every now and then that
every organization eventually needs fresh ideas and leadership.
When that time comes,
train your successors and move on as gracefully as you can.
They will undoubtedly do things you wouldn't have,
but few things in life are as satisfying as
watching something you helped build take on a life of its own.
Celebrate that---you won't have any trouble finding something else to keep you busy.

\seclbl{Exercises}{s:community-exercises}

Several of these exercises are taken from~\cite{Brow2007}.

\exercise{What Kind of Community?}{individual}{15}

Re-read the description of the four types of communities
and decide which one(s) your group is or aspires to be.

\exercise{People You May Meet}{small groups}{30}

As an organizer,
part of your job is sometimes to help people find a way to contribute despite themselves.
In small groups,
pick three of the people below
and discuss how you would help them become a better contributor to your organization.

\begin{description}

\item[Anna]
  knows more about every subject than everyone else put together---at least,
  she thinks she does.
  No matter what you say,
  she'll correct you;
  no matter what you know, she knows better.

\item[Catherine]
  has so little confidence in her own ability
  that she won't make any decision,
  no matter how small,
  until she has checked with someone else.

\item[Frank]
  enjoys knowing things that other people don't.
  He can work miracles,
  but when asked how he did it,
  he'll grin and say,
  ``Oh, I'm sure you can figure it out.''

\item[Hediyeh]
  is quiet.
  She never speaks up in meetings,
  even when she knows other people are wrong.
  She might contribute to the mailing list,
  but she's very sensitive to criticism
  and always backs down instead of defending her point.

\item[Ken]
  takes advantage of the fact that most people would rather shoulder his share of the work
  than complain about him.
  The frustrating thing is that he's so plausible when someone finally does confront him.
  ``There have been mistakes on all sides,''
  he says,
  or, ``Well, I think you're nit-picking.''

\item[Melissa]
  means well,
  but somehow something always comes up,
  and her tasks are never finished until the last possible moment.
  Of course,
  that means that everyone who is depending on her can't do their work
  until \emph{after} the last possible moment{\ldots}

\item[Raj]
  is rude.
  ``It's just the way I talk,'' he says.
  ``If you can't hack it, go find another team.''
  His favorite phrase is, ``That's stupid,''
  and he uses an obscenity in every second sentence.

\end{description}

\exercise{Values}{small groups}{45}

Answer these questions on your own,
then compare your answers with others'.

\begin{enumerate}

\item
  What are the values your organization expresses?

\item
  Are these the values you want the organization to express?

\item
  If not, what values would you like it to express?

\item
  What are specific behaviors that demonstrate those values?

\item
  What behaviors would demonstrate the opposite of those values?
\end{enumerate}

\exercise{Meeting Procedures}{small groups}{30}

Answer these questions on your own,
then compare your answers with others'.

\begin{enumerate}

\item
  How are your meetings run?

\item
  Is this how you want your meetings to be run?

\item
  Are the rules for running meetings explicit or just assumed?

\item
  Are these the rules you want?

\item
  Who is eligible to vote or make decisions?

\item
  Is this who you want to be vested with decision-making authority?

\item
  Do you use majority rule, make decisions by consensus, or something else?

\item
  Is this the way you want to make decisions?

\item
  How do people in a meeting know when a decision has been made?

\item
  How do people who weren't at a meeting know what decisions were made?

\item
  Is this working for your group?

\end{enumerate}

\exercise{Size}{small groups}{20}

Answer these questions on your own,
then compare your answers with others'.

\begin{enumerate}

\item
  How big is your group?

\item
  Is this the size you want for your organization?

\item
  If not, what size would you like it to be?

\item
  Do you have any limits on the size of membership?

\item
  Would you benefit from setting such a limit?

\end{enumerate}

\exercise{Becoming a Member}{small groups}{45}

Answer these questions on your own,
then compare your answers with others'.

\begin{enumerate}

\item
  How does someone join your group?

\item
  How well does this process work?

\item
  Are there membership dues?

\item
  Are people required to agree to any rules of behavior upon joining?

\item
  Are these the rules for behavior you want?

\item
  How does a newcomer find out what needs to be done?

\item
  How well does this process work?
  
\end{enumerate}

\exercise{Staffing}{small groups}{30}

Answer these questions on your own,
then compare your answers with others'.

\begin{enumerate}

\item
  Do you have paid staff in your organization
  or is everyone a volunteer?

\item
  Should you have paid staff?

\item
  Do you want/need more or less staff?

\item
  What do the staff members do?

\item
  Are these the primary roles and functions that you need staff to fill?

\item
  Who supervises your staff?

\item
  Is this the supervision process that you want for your group?

\item
  What is your staff paid?

\item
  Is this the right salary to get the needed work done?

\end{enumerate}

\exercise{Money}{small groups}{30}

Answer these questions on your own,
then compare your answers with others'.

\begin{enumerate}

\item
  Who pays for what?

\item
  Is this who you want to be paying?

\item
  Where do you get your money?

\item
  Is this how you want to get your money?

\item
  If not, do you have any plans to get it another way?

\item
  If so, what are they?

\item
  Who is following up to make sure that happens?

\item
  How much money do you have?

\item
  How much do you need?

\item
  What do you spend most of your money on?

\item
  Is this how you want to spend your money?

\end{enumerate}

\exercise{Borrowing Ideas}{whole class}{15}

Many of my ideas about how to build a community
have been shaped by my experience in open source software development.
\cite{Foge2005} (which is \hreffoot{http://producingoss.com/}{available online})
is a good guide to what has and hasn't worked for those communities,
and the \hreffoot{https://opensource.guide/}{Open Source Guides site}
has a wealth of useful information as well.
Choose one section of the latter,
such as ``Finding Users for Your Project''
or ``Leadership and Governance,''
and give a two-minute presentation to the group of one idea from it
that you found useful or that you strongly disagreed with.

\exercise{Who Are You?}{small groups}{20}

The National Oceanic and Atmospheric Administration (NOAA)
has a short, useful, and amusing guide to
\hreffoot{https://coast.noaa.gov/ddb/story\_html5.html}{dealing with disruptive behaviors}.
It categorizes those behaviors under labels like ``talkative,'' ``indecisive,'' and ``shy,''
and outlines strategies for handling each.
In groups of 3--6,
read the guide and decide which of these descriptions best fits you.
Do you think the strategies described for handling people like you are effective?
Are other strategies equally or more effective?

\exercise{Creating Lessons Together}{small groups}{30}

One of the keys to the success of \hreffoot{http://carpentries.org}{the Carpentries}
is their emphasis on building and maintaining lessons collaboratively~\cite{Wils2016,Deve2018}.
Working in groups of 3--4:

\begin{enumerate}

\item
  Pick a short lesson you have all used.

\item
  Do a careful review to create a unified list of suggestions for improvements.

\item
  Offer those suggestions to the lesson's author.

\end{enumerate}

\exercise{Are You Crispy?}{individual}{10}

\hreffoot{https://mailchi.mp/d54702d0a790/take-my-horse-to-the-sand-hill-road}{Johnathan Nightingale wrote}:

\begin{quote}
  When I worked at Mozilla,
  we used the term "crispy" to refer to the state right before burnout.
  People who are crispy aren't fun to be around.
  They are curt.
  They are itching for a fight they can win.
  They cry without much warning.
  {\ldots}we would recognize crispiness in our colleagues and take care of each other
  [but] it is an ugly thing that we saw it so much that we had a whole cultural process around it.
\end{quote}

\noindent
Responde ``sí'' o ``no'' a cada una de las siguientes preguntas.
¿Qué tan cerca estás de burning out?

\begin{itemize}
\item ¿Te has vuelto cínica/o o crítica/o en el trabajo?
\item ¿Tienes que arrastrarte al trabajo o tienes problemas para comenzar a trabajar?
\item ¿Te has vuelto irritable o impaciente con tus compañeros de trabajo?
\item ¿Te resulta difícil concentrarte?
\item ¿No logras satisfacción de tus logros?
\item ¿Estás usando comida, drogas o alcohol para sentirte mejor o simplemente no sentir?
\end{itemize}
