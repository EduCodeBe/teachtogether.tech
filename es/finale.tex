\chapter{¿Por qué enseño?}\label{s:finale}

Cuando comencé a ser voluntario en la Universidad de Toronto,
algunos de mis alumnos me preguntaron por qué iba a enseñar gratis.
Esta fue mi respuesta:

\begin{quote}

Cuando tenía tu edad,
pensaba que las universidades existían para enseñarle a la gente a aprender.
Más tarde,
en la escuela de posgrado,
pensé que las universidades se dedicaban investigar y crear nuevos conocimientos.
Sin embargo, ahora que tengo más de cuarenta años, 
me dí cuenta que lo que realmente te estamos enseñando es 
como hacerte cargo del mundo,
porque vas a tener que hacerlo quieras o no.

Mis padres tienen setenta años.
Ya no manejan el mundo;
son las personas de mi edad quienes aprueban leyes
y toman decisiones de vida o muerte en los hospitales.
Y tan aterrador como es, 
\emph{nosotros} somos los adultos ahora.

Dentro de veinte años,
nosotros iremos hacia la jubilación y \emph{tu} estarás a cargo.
Eso puede parecer mucho tiempo cuando tienes diecinueve años,
pero respira tres veces y se pasó.
Es por eso que te damos problemas cuyas respuestas no se pueden encontrar en las notas del año pasado.
Por eso te ponemos en situaciones donde
donde tienes que decidir que hay que hacer ahora,
qué se puede dejar para más tarde
y qué simplemente puedes ignorar.
Es porque si no aprendes cómo hacer estas cosas ahora,
no estarás listo para hacerlo cuando sea necesario.

\end{quote}

Todo era verdad,
pero no es toda la historia.
No quiero que la gente haga del mundo un lugar mejor para poder retirarme cómodamente.
Quiero que lo hagan porque es la mayor aventura de nuestro tiempo.
Hace ciento cincuenta años,
la mayoría de las sociedades practicaban la esclavitud.
Hace cien años,
mi abuela \hreffoot{https://en.wikipedia.org/wiki/The\_Famous\_Five\_(Canada)}{no era legalmente una persona} en Canada.
En el año que nací,
la mayoría de las personas del mundo sufría bajo algún gobierno totalitario,
y los jueces todavía ordenaban terapia de electroshock para ``curar'' a los homosexuales.
Todavía hay muchas cosas mal en el mundo,
pero mira cuántas más opciones tenemos nosotros comparadas con las de nuestros abuelos.
Mira cuántas cosas más podemos saber, ser y disfrutar
porque finalmente nos estamos tomando en serio la Regla de Oro.

Hoy soy menos optimista que entonces.
Cambio climático,
extinción masiva,
capitalismo de vigilancia,
desigualdad en una escala que no hemos visto en un siglo,
El resurgimiento del nacionalismo racista:
mi generación lo ha visto todo y se encogió de hombros.
La factura de nuestra cobardía, letargo y avaricia no se pagará hasta que mi hija crezca,
pero \emph{llegará},
y para cuando lo haga, no habrá soluciones fáciles para estos problemas
(y posiblemente no haya soluciones en absoluto).

Así que por eso enseño hoy:
estoy enojado.
Estoy enojado porque tu sexo y tu color y la riqueza y conexiones de tus padres
no debería contar más que cuán inteligente, honesto/a o trabajador/a eres.
Estoy enojado porque convertimos Internet en una cloaca.
Estoy enojado porque los nazis están en marcha una vez más
y multimillonarios juegan con cohetes mientras el planeta se derrite.
Estoy enojado,
entonces enseño,
porque el mundo solo mejora cuando enseñamos a las personas cómo mejorarlo.

En su ensayo de 1947 ``¿Por qué escribo?'',
\hreffoot{http://www.resort.com/~prime8/Orwell/whywrite.html}{George Orwell} escribió:
\index{Orwell, George}

\begin{quote}

  En una época pacífica, podría haber escrito libros ornamentados o meramente descriptivos,
  y podría haber permanecido casi inconsciente de mis lealtades políticas.
  Tal como están las cosas, me he visto obligado a convertirme en una especie de panfleto{\ldots}
  Cada línea de trabajo serio que he escrito desde 1936 ha sido escrita,
  directa o indirectamente,
  contra el totalitarismo{\ldots}
  Me parece una tontería,
  en un período como el nuestro,
  pensar que se puede evitar escribir sobre tales temas.
  Todos escriben de ellos de una manera u otra.
  Es simplemente una cuestión sobre qué lado vas a elegir.

\end{quote}

\noindent
Reemplaza ``escribir'' con ``enseñar'' y tendrás la razón por la que hago lo que hago.

\vspace{\baselineskip}

\noindent
Gracias por leer --- Espero que podamos enseñar juntos algún día.
Hasta entonces:

\begin{center}

Comienza donde estás. \\
Usa lo que tienes. \\
Ayuda a quien puedas.

\end{center}
