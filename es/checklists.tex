\chapter{Listas de verificación y plantillas}\label{s:checklist}

\cite{Gawa2007} hizo popular la idea de que usar listas de verificación puede salvar vidas,
y estudios más recientes apoyan su efectividad ~\cite{Avel2013,Urba2014,Rams2019}.
Encontramos las listas de verificación útiles,
en particular cuando hay docentes que recién se incorporan al equipo.
Los ejemplos a continuación pueden servirte como material inicial a partir del cual desarrollar tus propias listas de verificación.

\seclbl{Enseñando a evaluar}{s:checklists-teacheval}

Esta rúbrica fue diseñada para evaluar lo enseñado durante 5 a 10 minutos
con diapositivas, programación en vivo o una combinación de ambas estrategias.
Valora cada ítem como ``sí,'' ``Más o menos,'' ``No,'' or ``No corresponde (N/A).''

\noindent
\begin{longtable}{p{.25\textwidth}p{.75\textwidth}}

  Inicio
  & Presente (usa N/A para otras respuestas) \\
  & Adecuada duración (10 a 30 segundos) \\
  & Se presenta \\
  & Presenta el tema que se trabajará \\
  & Describe los requisitos \\
  \\ [-1.5ex] \hline \\ [-1.5ex]

  Contenido
  & Objetivos claros/narrativa fluida \\
  & Lenguaje inclusivo \\
  & Ejemplos y tareas reales \\
  & Enseña buenas prácticas/utiliza el idioma del código \\
  & Señala un camino intermedio entre la Escila de la jerga y la Caribdis de la sobresimplificación \\
  \\ [-1.5ex] \hline \\ [-1.5ex]

  Dando la lección
  & Voz clara y entendible (usa ``Más o menos'' o ``No'' para acentos muy marcados) \\
  & Ritmo: ni muy rápido ni muy lento, no hace pausas largas o se interrumpe, no aparenta estar leyendo sus notas \\
  & Seguridad: no se pierde en el pozo de alquitrán de la incertidumbre ni tampoco en las colinas de estiércol de la condescendencia \\
  \\ [-1.5ex] \hline \\ [-1.5ex]

  Diapositivas
  & Usa diapositivas (utiliza N/A para otras respuestas) \\
  & Diapositivas y discurso se complementan uno al otro (programación dual) \\
  & Fuentes y colores legibles/sin bloques de texto abrumadores por su tamaño\\
  & Pantalla cambia frecuentemente (algo cada 30 segundos) \\
  & Adecuado uso de figuras \\
  \\ [-1.5ex] \hline \\ [-1.5ex]

  Programación en vivo
  & Usa programación en vivo (valora N/A para otras respuestas) \\
  & Código y discurso se complementan uno al otro\\
  & Fuentes y colores legibles/adecuada cantidad de código en pantalla \\
  & Uso de herramientas de forma adecuada \\
  & Resalta elementos clave del código \\
  & Analiza los errores \\
  \\ [-1.5ex] \hline \\ [-1.5ex]

  Cierre
  & Presente (valora N/A para otras respuestas) \\
  & Adecuada duración (10 a 30 segundos) \\
  & Resume puntos clave \\
  & Presenta un esquema general de los próximos pasos \\
  \\ [-1.5ex] \hline \\ [-1.5ex]

  En general
  & Puntos claramente conectados/flujo lógico \\
  & Hace que el tema sea interesante (i.e.\ no aburrido) \\
  & Comprende el tema \\

\end{longtable}

\seclbl{Evaluación del grupo docente}{s:checklists-teameval}

Esta rúbrica fue diseñada para evaluar el desempeño de individuos dentro de un grupo. 
Los ejemplos a continuación pueden servirte como material inicial a partir del cual desarrollar tus propias rúbricas. 
Valora cada ítem como ``sí,'' ``Más o menos,'' ``No,'' or ``No corresponde (N/A).''

\noindent
\begin{longtable}{p{.25\textwidth}p{.75\textwidth}}

  Comunicación
  & Escucha atentamente y sin interrumpir \\
  & Aclara lo que se ha dicho para asegurar la comprensión \\
  & Articula ideas en forma clara y concisa \\
  & Argumenta adecuadamente sus ideas \\
  & Obtiene el apoyo de otros miembros del equipo \\
  \\ [-1.5ex] \hline \\ [-1.5ex]

  Toma de decisiones
  & Analiza los problemas desde diferentes puntos de vista \\
  & Aplica lógica para resolver problemas \\
  & Propone soluciones basadas en hechos y no en ``corazonadas'' o intuición \\
  & Invita a los miembros del equipo a proponer nuevas ideas \\
  & Genera nuevas ideas \\
  & Acepta cambios \\
  \\ [-1.5ex] \hline \\ [-1.5ex]

  Colaboración
  & Reconoce los problemas que elequipo necesita enfrentar y resolver \\
  & Trabaja para hallar soluciones que sean aceptables para todas las partes involucradas \\
  & Comparte el crédito del éxito con otros miembros del equipo \\
  & Promueve la participación entre todos los miembros del equipo \\
  & Acepta la crítica abiertamente y sin ``ponerse a la defensiva'' \\
  & Coopera con el equipo \\
  \\ [-1.5ex] \hline \\ [-1.5ex]

  Autogestión
  & Monitorea sus avances para asegurar que se alcancen los objetivos \\
  & Le da máxima prioridad a obtener resultados \\
  & Define tareas prioritarias para los encuentros de trabajo \\
  & Promueve que otros miembros del equipo manifiesten sus opiniones, incluso si no coinciden con las propias \\
  & Mantiene la atención durante la reunión \\
  & Usa eficientemente el tiempo de reunión \\
  & Sugiere formas de trabajar en las reuniones \\

\end{longtable}

\seclbl{Organización de eventos}{s:checklists-events}

Las listas de verificación a continuación pueden usarse antes, durante y después de un evento.

\subsection*{Programar el evento}

\begin{itemize}

\item
  Decidir si será presencial,
  virtual para un lugar,
  o virtual para más de un lugar.

\item
  Conversar con la/el disertante? sobre sus expectativas
  y asegurarse que están de acuerdo en cuanto a quién cubrirá los costos de traslado.

\item
  Definir quiénes podrán participar:
  ¿será el evento abierto a todas las personas?
  ¿restringido a miembros de una organización?
  ¿una situación intermedia?

\item
  Organizar quiénes serán docentes.

\item
  Organizar el espacio, incluyendo *breakout rooms* si fuera necesario.

\item
  Definir la fecha.
  Si fuera presencia, reservar lo relativo al viaje.

\item
  Conseguir nombres y direccione de e-mail de participantes a través de la/el disertante.

\item
  Asegurarse que la totalidad de las y los participantes esté registrada.

\end{itemize}

\subsection*{Construcción del evento}

\begin{itemize}

\item
  Crea una página web con los detalles del taller,
  que incluya fecha,
  lugar,
  y lo que las y los participantes deben traer consigo.

\item
  Confirma las necesidades espciales de las/los participantes.

\item
  Si el evento será virtual prueba el modo de videoconferencia, dos veces.

\item
  Asegúrate que las/los participantes tengan acceso a internet.

\item
  Crea un espacio para compartir apuntes y soluciones a los ejercicios (p.ej.\ un documento Google Doc).

\item
  Establece contacto con las/los asistentes por e-mail con un mensaje de bienvenida que contenga
  el link a la página del taller,
  lecturas sobre la temática,
  la descripción de la configuración que deba hacer,
  una lista e los elemtnos requeridos para el taller,
  y un mecanismo para establecer contacto con la/el disertante o docente durante el día.

\end{itemize}

\subsection*{Al comienzo del evento}

\begin{itemize}

\item
  Recuerda a las y los asistentes el código de conducta.

\item
  Toma lista
  y crea una lista de nombres para pegar en la página compartida para tomar notas.

\item
  Reparte *post its*.

\item
  Asegúrate que tengan acceso a internet.
  
\item
  Asegúrate que pueden acceder a la página compartida.

\item
  Registra información relevante sobre la identificación de las/los asistentes en sus perfiles online.

\end{itemize}

\subsection*{Al finalizar el evento}

\begin{itemize}

\item
  Actualiza la lista de participantes.

\item
  Lleva un registro del feedbak dado por las/los participantes.

\item
  Haz una copia de la página compartida.

\end{itemize}

\subsection*{Equipo de viaje}

Aquí algunas cosas que las/los docentes llevan consigo a los talleres:

\begin{longtable}{p{0.45\textwidth}p{0.45\textwidth}}

postits y caramelos para suavizar la garganta \\
zapatos cómodos y pequeña libreta de notas \\
adaptador de corriente eléctrica de repuesto y camisa de repuesto \\
desodorante y adaptadores para video \\
pegatinas (*stickers*) para computadoras y tus notas (impresas o en una tableta) \\
barrita de cereal o similar y antiácido (problema de comer al paso) \\
tarjeta de presentación y anteojos/lentes de contacto de repuesto \\
libreta y bolígrafo, y puntero láser \\
vaso térmico para té/café y marcadores de pizarra adicionales \\
cepillo de dientes o enjuague bucal y toallitas húmedas descartables (puede volcarse algo encima de tu ropa) \\

\end{longtable}

Al viajar
muchas/os docentes llevan además zapatos deportivos, traje de baño, mat de yoga
o el material que necesiten para hacer actividad física.
También una conexión WiFi portátil por si la de la habitación no funciona,
y alguna memoria USB con los instaladores del software que las/los estudiantes aprenderán.

\seclbl{Diseño de lecciones}{s:checklists-design}

Esta sección resume el diseño de lecciones por el método hacia atrás o *backward* en inglés,
que fue desarrolado independientemente por ~\cite{Wigg2005,Bigg2011,Fink2013}.
Propone una progresión paso a paso
para ayudarte a pensar en qué hacer en cada uno y en el orden adecuado
y proporciona ejercicios breves espaciados
para que puedas reorientar o redirigir tu esfuerzo sin demasiadas sorpresas desagradables.

Del paso 2 en adelante será considerado en tu lección final
por lo que no se trata de un desperdicio de esfuerzo:
como se describió en el \chapref{s:process},
construir ejercicios de práctica desde el comienzo te ayuda a asegurarte que
todo lo que preguntes a las/los estudiantes contribuirá a los objetivos de la lección
y que todo lo que necesitan saber está cubierto.

Los pasos se describen en orden creciente de detalle
pero el proceso en sí es siempre iterativo.
Con frecuencia volverás a revisar tus respuestas en trabajos anteriores
a medida que resuelvas preguntas más avanzadas
o te des cuenta que tu primera idea sobre cómo resolver algo no iba a funcionar de la manera que pensaste originalmente.

\subsection*{¿Para quién es esta lección?}

Crea algunas/os estudiantes tipo (\secref{s:process-personas})
o (mejor aún) elige de entre los que tú y tus colegas han creado para uso general.
Cada estudiante tipo debe tener:

\begin{enumerate}

\item
  un contexto general,

\item
  lo que ya sabe,

\item
  lo que cree que quiere saber y

\item
  qué necesidades especiales tiene.

\end{enumerate}

~\\
\noindent
\textbf{Ejercicio breve:} resumen breve de a quién estas intentando ayudar.

\subsection*{¿Cuál es la idea principal?}

Responde tres o cuatro de las preguntas a continuación solo enumerando elementos
para ayudarte a descifrar el enfoque de la lección.
No necesitas responder todas las preguntas,
y puedes plantear y responder otras preguntas si creer que ayudarán,
pero debes incluir sí o sí un par de respuestas a la primera pregunta.
Además, en esta etapa puedes crear un mapa conceptual (\secref{s:memory-concept-maps}).

\begin{itemize}

\item
  ¿Qué problemas aprenderán a resolver?

\item
  ¿Cuáles conceptos y técnicas aprenderán?

\item
  ¿Cuáles herramientas tecnológicas, paquetes y funciones usarán?

\item
  ¿Qué terminos de la jerga definirás?

\item
  ¿Qué analogías usarás para explicar conceptos?

\item
  ¿Qué errores o conceptos equivocados esperas encontrar?

\item
  ¿Cuáles grupos de datos utilizarás?

\end{itemize}

~\\
\noindent
\textbf{Ejercicio breve}
a rough scope for the lesson.
Share this with a colleague---a little bit of feedback at this point
can save hours of wasted effort later on.

\subsection*{What will learners do along the way?}

Make the goals in Step 2 firmer by writing full descriptions of
a couple of exercises that learners will be able to do toward the end of the lesson.
Doing this is analogous to \hreffoot{https://en.wikipedia.org/wiki/Test-driven\_development}{test-driven development}:
rather than working forward from a (probably ambiguous) set of learning objectives,
work backward from concrete examples of where you want your learners to end up.
Doing this also helps uncover technical requirements
that might otherwise not be found until uncomfortably late.

To complement the full exercise descriptions,
write brief point-form descriptions of one or two exercises per lecture hour
to show how quickly you expect learners to progress.
Again,
these serve as a good reality check on how much you're assuming
and help uncover technical requirements.
One way to create these ``extra'' exercises
is to make a point-form list of the skills needed to solve the major exercises
and create an exercise that targets each.

~\\
\noindent
\textbf{Deliverable:} 1--2 fully explained exercises
that use the skills people are to learn,
plus half a dozen point-form exercise outlines.
Include complete solutions
so that you can make sure the software you want learners to use actually works.

\subsection*{How are concepts connected?}

Put the exercises you have created in a logical order
and then derive a point-form lesson outline from them.
The outline should have 3--4 bullet points for each hour
with a formative assessment of some kind for each.
It is common to change assessments in this stage
so that they can build on each other.

~\\
\noindent
\textbf{Deliverable:} a lesson outline.
You are likely to discover things you forgot to list earlier during this stage,
so don't be surprised if you have to double back a few times.

\subsection*{Lesson overview}

You can now write a lesson overview with:

\begin{itemize}

\item
  a one-paragraph description (i.e.\ a sales pitch to learners);

\item
  half a dozen learning objectives; and

\item
  a summary of prerequisites.

\end{itemize}

Doing this earlier often wastes effort,
since material is usually added, cut, or moved around in earlier steps.

~\\
\noindent
\textbf{Deliverable:}
course description,
learning objectives,
and prerequisites.

\seclbl{Pre-Assessment Questionnaire}{s:checklists-preassess}

This questionnaire helps teachers gauge the prior programming knowledge
of participants in an introductory JavaScript workshop.
The questions and answers are concrete,
and the whole thing is short so that respondents won't find it intimidating.

\begin{enumerate}

\item
  Which of these best describes
  your experience with programming in general?

  \begin{itemize}
    
  \item
    I have none.
    
  \item
    I have written a few lines now and again.
    
  \item
    I have written programs for my own use that are a couple of
    pages long.
    
  \item
    I have written and maintained larger pieces of software.\\
    
  \end{itemize}

\item
  Which of these best describes
  your experience with programming in JavaScript?

  \begin{itemize}
    
  \item
    I have none.
    
  \item
    I have written a few lines now and again.
    
  \item
    I have written programs for my own use that are a couple of
    pages long.
    
  \item
    I have written and maintained larger pieces of software.\\
    
  \end{itemize}

\item
  Which of these best describes how easily you could write a program
  in any language
  to find the largest number in a list?

  \begin{itemize}
    
  \item
    I wouldn't know where to start.
    
  \item
    I could struggle through by trial and error with a lot of web
    searches.
    
  \item
    I could do it quickly with little or no use of external help.\\
    
  \end{itemize}

\item
  Which of these best describes
  how easily you could write a JavaScript program
  to find and capitalize all of the titles in a web page?

  \begin{itemize}
    
  \item
    I wouldn't know where to start.
    
  \item
    I could struggle through by trial and error with a lot of web
    searches.
    
  \item
    I could do it quickly with little or no use of external help.\\
    
  \end{itemize}

\item
  What do you want to know or be able to do after this class
  that you don't know or can't do right now?

\end{enumerate}
