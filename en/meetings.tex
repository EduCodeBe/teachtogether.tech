\chapter{Meetings, Meetings, Meetings}\label{s:meetings}

Most people are really bad at meetings:
they don't make an agenda,
they don't take minutes,
they waffle on or wander off into irrelevancies,
they say something banal or repeat what others have said
just so that they'll have said something,
and they hold side conversations
(which pretty much guarantees that the meeting will be a waste of time).
Knowing how to run a meeting efficiently
is a core skill for anyone who wants to get things done;
knowing how to take part in someone else's meeting is just as important
(though it gets far less attention---as a colleague once said,
everyone offers leadership training,
but nobody offers followership training).

The most important rules for making meetings efficient are not secret,
but are rarely followed:

\begin{description}

\item[Decide if there actually needs to be a meeting.]
  If the only purpose is to share information,
  send a brief email instead.
  Remember,
  you can read faster than anyone can speak:
  if someone has facts for the rest of the team to absorb,
  the most polite way to communicate them is to write them up.

\item[Write an agenda.]
  If nobody cares enough about the meeting to write a point-form list
  of what's to be discussed,
  the meeting probably doesn't need to happen.

\item[Include timings in the agenda.]
  Agendas can also help you prevent early items stealing time from later ones
  if you include the time to be spent on each item in the agenda.
  Your first estimates with any new group will be wildly optimistic,
  so revise them upward for subsequent meetings.
  However,
  you shouldn't plan a second or third meeting
  because the first one ran over-time:
  instead,
  try to figure out why you're running over and fix the underlying problem.

\item[Prioritize.]
  Every meeting is a micro-project,
  so work should be prioritized in the same way that it is for other projects:
  things that will have high impact but take little time should be done first,
  and things that will take lots of time but have little impact should be skipped.

\item[Make one person responsible for keeping things moving.]
  One person should be tasked with keeping items to time,
  chiding people who are checking email or  having side conversations,
  asking people who are talking too much to get to the point,
  and inviting people who aren't talking to express an opinion.
  This person should \emph{not} do all the talking;
  in fact,
  whoever is in charge will talk less in a well-run meeting
  than most other participants.

\item[Require politeness.]
  No one gets to be rude,
  no one gets to ramble,
  and if someone goes off topic,
  it is both the moderator's right and responsibility to say,
  ``Let's discuss that elsewhere.''

\item[No interruptions.]
  Participants should raise a finger or put up a sticky note
  if they want to speak next.
  If the person speaking doesn't notice them,
  the moderator should.

\item[No technology]
  unless it's required for accessibility reasons.
  Insist that everyone put their phones, tablets, and laptops into politeness mode
  (i.e.\ close them).

\item[Record minutes.]
  Someone other than the moderator should take point-form notes about
  the most important pieces of information that were shared,
  every decision that was made,
  and every task that was assigned to someone.

\item[Take notes.]
  While other people are talking,
  participants should take notes of questions they want to ask or points they want to make.
  (You'll be surprised how smart it makes you look when it's your turn to speak.)

\item[End early.]
  If your meeting is scheduled for 10:00-11:00,
  you should aim to end at 10:50 to give people time to visit the bathroom
  on their way to where they need to go next.

\end{description}

As soon as the meeting is over,
email the minutes to everyone or post them on the web:

\begin{description}

\item[People who weren't at the meeting can keep track of what's going on.]
  A web page or email message is a much more efficient way to catch up
  than asking a team mate what you missed.

\item[Everyone can check what was actually said or promised.]
  More than once,
  I have looked over the minutes of a meeting I was in
  and thought, ``Did I say that?''
  or, ``Wait a minute, I didn't promise to have it ready then!''
  Accidentally or not,
  people will often remember things differently;
  writing it down gives team members a chance to correct mistakes,
  which can save a lot of anguish later on.

\item[People can be held accountable at subsequent meetings.]
  There's no point making lists of questions and action items
  if you don't follow up on them later.
  If you're using some kind of issue tracking system,
  create an issue for each new question or task right after the meeting
  and update those that are being carried forward,
  then start each meeting by going through a list of those issues.

\end{description}

\cite{Brow2007,Broo2016,Roge2018} have lots of advice on running meetings.
In my experience,
an hour of training on how to be a moderator
is one of the best investments you will ever make.

\begin{aside}{Sticky Notes and Interruption Bingo}
  Some people are so used to the sound of their own voice
  that they will insist on talking half the time
  no matter how many other people are in the room.
  To combat this,
  give everyone three sticky notes at the start of the meeting.
  Every time they speak,
  they have to take down one sticky note.
  When they're out of notes,
  they aren't allowed to speak until everyone has used at least one,
  at which point everyone gets all of their sticky notes back.
  This ensures that nobody talks more than three times as often as
  the quietest person in the meeting,
  and completely changes the dynamics of most groups:
  people who have given up trying to be heard because they always get trampled
  suddenly have space to contribute,
  and the overly-frequent speakers quickly realize just how unfair they have been\footnote{
    I certainly did when this was done to me{\ldots}
  }.

  Another technique is interruption bingo.
  Draw a grid and label the rows and columns with participants' names.
  Add a tally mark to the appropriate cell
  each time someone interrupts someone else,
  and take a moment to share the results halfway through the meeting.
  In most cases,
  you will see that one or two people are doing all of the interrupting,
  often without being aware of it.
  That alone is often enough to get them to throttle back.
  (Note that this technique is intended to manage interruptions,
  not speaking time:
  it may be appropriate for people with more knowledge of a subject
  to speak about it more often in a meeting,
  but it is never appropriate to repeatedly cut people off.)
\end{aside}

\seclbl{Martha's Rules}{s:meetings-marthas-rules}

Organizations all over the world run their meetings according to
\hreffoot{https://en.wikipedia.org/wiki/Robert\%27s\_Rules\_of\_Order}{Robert's Rules of Order},
but they are far more formal than most small projects require.
A lightweight alternative known as ``Martha's Rules''
may be much better for consensus-based decision making~\cite{Mina1986}:

\begin{enumerate}

\item
  Before each meeting,
  anyone who wishes may sponsor a proposal by sharing it with the group.
  Proposals must be filed at least 24 hours before a meeting in order to be considered at that meeting,
  and must include:
  \begin{itemize}
  \item a one-line summary;
  \item the full text of the proposal;
  \item any required background information;
  \item pros and cons; and
  \item possible alternatives
  \end{itemize}
  Proposals should be at most two pages long.

\item
  A quorum is established in a meeting if half or more of voting members are present.

\item
  Once a person has sponsored a proposal,
  they are responsible for it.
  The group may not discuss or vote on the issue unless the sponsor or their delegate is present.
  The sponsor is also responsible for presenting the item to the group.

\item
  After the sponsor presents the proposal,
  a preliminary vote is cast for the proposal prior to any discussion:
  \begin{itemize}
  \item Who likes the proposal?
  \item Who can live with the proposal?
  \item Who is uncomfortable with the proposal?
  \end{itemize}
  Preliminary votes can be done thumb up, thumb sideways, or thumb down (in person)
  or by typing +1, 0, or -1 into online chat (in virtual meetings).

\item
  If all or most of the group likes or can live with the proposal,
  it is immediately moved to a formal vote with no further discussion.

\item
  If most of the group is uncomfortable with the proposal,
  it is postponed for further rework by the sponsor.

\item
  If some members are uncomfortable they can briefly state their objections.
  A timer is then set for a brief discussion moderated by the facilitator.
  After ten minutes or when no one has anything further to add (whichever comes first),
  the facilitator calls for a yes-or-no vote on the question:
  ``Should we implement this decision over the stated objections?''
  If a majority votes ``yes'' the proposal is implemented.
  Otherwise, the proposal is returned to the sponsor for further work.

\end{enumerate}

\seclbl{Online Meetings}{s:meetings-online}

\hreffoot{https://chelseatroy.com/2018/03/29/why-do-remote-meetings-suck-so-much/}{Chelsea Troy's discussion}
of why online meetings are often frustrating and unproductive
makes an important point:
in most online meetings,
the first person to speak during a pause gets the floor.
The result?
``If you have something you want to say,
you have to stop listening to the person currently speaking
and instead focus on when they're gonna pause or finish
so you can leap into that nanosecond of silence and be the first to utter something.
The format{\ldots}encourages participants who want to contribute to say more and listen less.''

The solution is to run a text chat beside the video conference
where people can signal that they want to speak,
The moderator then selects people from the waiting list.
If the meeting is large or argumentative,
have everyone mute themselves
and only allow the moderator to unmute people.

\seclbl{The Post Mortem}{s:meetings-post-mortem}

Every project should end with a post mortem
in which participants reflect on what they just accomplished
and what they could do better next time.
Its aim is \emph{not} to point the finger of shame at individuals,
although if that has to happen,
the post mortem is the best place for it.

A post mortem is run like any other meeting
with a few additional guidelines~\cite{Derb2006}:

\begin{description}

\item[Get a moderator who wasn't part of the project]
  and doesn't have a stake in it.

\item[Set aside an hour, and only an hour.]
  In my experience,
  nothing useful is said in the first ten minutes of anyone's first post mortem,
  since people are naturally a bit shy about praising or damning their own work.
  Equally,
  nothing useful is said after the first hour:
  if you're still talking,
  it's probably because one or two people
  have things they want to get off their chests
  rather than suggestions for improvements.

\item[Require attendance.]
  Everyone who was part of the project ought to be in the room for the post mortem.
  This is more important than you might think:
  the people who have the most to learn from the post mortem
  are often least likely to show up if the meeting is optional.

\item[Make two lists.]
  When I'm moderating,
  I put the headings ``Do Again'' and ``Do Differently'' on the board,
  then ask each person to give me one item for each list in turn
  without repeating anything that has already been said.

\item[Comment on actions rather than individuals.]
  By the time the project is done,
  some people may no longer be friends.
  Don't let this sidetrack the meeting:
  if someone has a specific complaint about another member of the team,
  require them to criticize a particular event or decision.
  ``They had a bad attitude'' does \emph{not} help anyone improve.

\item[Prioritize the recommendations.]
  Once everyone's thoughts are out in the open,
  sort them according to which are most important to keep doing
  and which are most important to change.
  You will probably only be able to tackle one or two from each list in your next project,
  but if you do that every time,
  your life will quickly get better.

\end{description}
