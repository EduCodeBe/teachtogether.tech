\chapter{Outreach}\label{s:outreach}

It's fashionable in tech circles
to disparage universities and government institutions as slow-moving dinosaurs,
but in my experience they are no worse than companies of similar size.
Your local school board, library, and your city councilor's office
may be able to offer space, funding, publicity,
connections with other groups that you may not have met yet,
and a host of other useful things;
getting to know them can help you solve or avoid problems in the short term
and pay dividends down the road.

\seclbl{Marketing}{s:outreach-marketing}

People with academic or technical backgrounds often think that
\gref{g:marketing}{marketing} is about spin and misdirection.
In reality,
it's about seeing things from other people's perspective,
understanding their wants and needs,
and explaining how you can help them---in short,
how to teach them.
This chapter will look at how to use ideas from the previous chapters
to get people to understand and support what you're doing.

The first step is to figure out what you are offering to whom,
i.e.\ what actually brings in the volunteers,
funding,
and other support you need to keep going.
The answer is often counter-intuitive.
For example,
most scientists think their papers are their product,
but it's actually their grant proposals,
because those are what brings in grant money~\cite{Kuch2011}.
Their papers are the advertising that persuades people to fund those proposals,
just as albums are now what persuades people to buy musicians' concert tickets and t-shirts.

Suppose that your group offers weekend programming workshops
to people who are re-entering the workforce after being away for several years.
If workshop participants can pay enough to cover your costs,
then they are your customers and the workshops are the product.
If,
on the other hand,
the workshops are free or the learners are only paying a token amount to cut the no-show rate,
then your actual product may be some mix of:

\begin{itemize}

\item
  your grant proposals;

\item
  the alumni of your workshops
  that the companies sponsoring you would like to hire;

\item
  the half-page summary of your workshops in the mayor's annual report to city council
  that shows how she's supporting the local tech sector;
  or

\item
  the personal satisfaction that volunteers get from teaching.

\end{itemize}

As with lesson design (\chapref{s:process}),
the first steps in marketing are
to create personas\index{learner persona}
of people who might be interested in what you're doing
and to figure out which of their needs you can meet.
One way to summarize the latter is to write \grefdex{g:elevator-pitch}{elevator pitches}{elevator pitch}
aimed at different personas.
A widely used template for these is:

\newpage
\begin{longtable}{ll}
  For 		& \emph{target audience} \\
  who 		& \emph{dissatisfaction with what's currently available} \\
  our 		& \emph{category} \\
  provide 	& \emph{key benefit}. \\
  Unlike 	& \emph{alternatives} \\
  our program 	& \emph{key distinguishing feature.}
\end{longtable}

\noindent
Continuing the weekend workshop example,
we could use this pitch for participants:

\begin{quote}

  For \emph{people re-entering the workforce after being away for several years}
  who \emph{still have family responsibilities},
  our \emph{introductory programming workshops}
  provide \emph{weekend classes with on-site childcare}.
  Unlike \emph{online classes},
  our program \emph{gives people a chance to meet others at the same stage of life}.

\end{quote}

\noindent
and this one for decision makers at companies that might sponsor the workshops:

\begin{quote}

  For \emph{companies that want to recruit entry-level software developers}
  that \emph{struggle to find mature candidates from diverse backgrounds}
  our \emph{introductory programming workshops}
  provide \emph{potential recruits}.
  Unlike \emph{college recruiting fairs},
  our program \emph{connects companies with a wide variety of candidates}.

\end{quote}

If you don't know why different potential stakeholders might be interested in what you're doing,
ask them.
If you do know,
ask them anyway:
answers can change over time,
and you may discover things you previously overlooked.

Once you have these pitches,
they should drive what you put on your website and in publicity material
to help people figure out as quickly as possible
if you and they have something to talk about.
(You probably \emph{shouldn't} copy them verbatim,
though:
many people in tech have seen this template so often that
their eyes will glaze over if they encounter it again.)

As you are writing these pitches,
remember that there are many reasons to learn how to program
(\secref{s:intro-exercises}).
A sense of accomplishment,
control over their own lives,
and being part of a community may motivate people more than money
(\chapref{s:motivation}).
They might volunteer to teach with you because their friends are doing it;
similarly,
a company may say that they're sponsoring classes for economically disadvantaged high school students
because they want a larger pool of potential employees further down the road,
but the CEO might actually be doing it simply because it's the right thing to do.

\seclbl{Branding and Positioning}{s:outreach-branding}

A \gref{g:brand}{brand} is someone's first reaction to a mention of a product;
if the reaction is ``what's that?'',
you don't have a brand (yet).
Branding is important because
people aren't going to help something they don't know about or don't care about.

Most discussion of branding today focuses on
how to build awareness online.
Mailing lists,
blogs,
and Twitter all give you ways to reach people,
but as the volume of misinformation increases,
people pay less attention to each individual interruption.
This makes \gref{g:positioning}{positioning} ever more important.
Sometimes called ``differentiation,''
it is what sets your offering apart from others---the ``unlike'' section of your elevator pitches.
When you reach out to people who are already familiar with your field,
you should emphasize your positioning,
since it's what will catch their attention.

There are other things you can do to help build your brand.
One is to use props
like a robot that one of your learners made from scraps she found around the house~\cite{Schw2013}
or the website another learner made for his parents' retirement home.
Another is to make a short video---no more than a few minutes long---that showcases
the backgrounds and accomplishments of your learners.
The aim of both is to tell a story:
while people always ask for data,
they believe and remember stories.

\begin{aside}{Foundational Myths}
  One of the most compelling stories a person or group can tell is
  why and how they got started.
  Are you teaching what you wish someone had taught you but didn't?
  Was there one particular person you wanted to help,
  and that opened the floodgates?
  If there isn't a section on your website starting, ``Once upon a time,''
  think about adding one.
\end{aside}

One crucial step is to make your organization findable in online searches.\index{findability!of organizations}
\cite{DiSa2014b} discovered that
the search terms that parents used for out-of-school computing classes
didn't actually find those classes,
and many other groups face similar challenges.
There's a lot of folklore about how to make things findable
(otherwise known as \gref{g:seo}{search engine optimization} or SEO);
given Google's near-monopoly powers and lack of transparency,
most of it boils down to trying to stay one step ahead of
algorithms designed to prevent people from gaming rankings.

Unless you're very well funded,
the best you can do is to search for yourself and your organization on a regular basis
and see what comes up,
then read \hreffoot{https://moz.com/learn/seo/on-page-factors}{these guidelines}
and do what you can to improve your site.
Keep \hreffoot{https://xkcd.com/773/}{this XKCD cartoon} in mind:
people don't want to know about your org chart or get a virtual tour of your site---they want your address,
parking information,
and some idea of what you teach,
when you teach it,
and how it's going to change their lives.

\begin{aside}{Not Everyone Lives Online}
  These examples assume people have access to the internet
  and that groups have money, materials, free time, and/or technical skills.
  Many don't---in fact,
  those serving economically disadvantaged groups almost certainly don't.
  (As Rosario Robinson says, ``Free works for those that can afford free.'')\index{Robinson, Rosario}
  Stories are more important than course outlines in those situations
  because they are easier to retell.
  Similarly,
  if the people you hope to reach are not online as often as you,
  then notice boards in schools,
  local libraries,
  drop-in centers,
  and grocery stores may be the most effective way to reach them.
\end{aside}

\seclbl{The Art of the Cold Call}{s:outreach-cold-call}

Building a website and hoping that people find it is easy;
calling people up or knocking on their door without any sort of prior introduction
is much harder.
As with standing up and teaching,
though,
it's a craft that can be learned.
Here are ten simple rules for talking people into things:

\begin{description}

\item[1: Don't.]
  If you have to talk someone into something,
  odds are that they don't really want to do it.
  Respect that:
  it's almost always better in the long run to leave some particular thing undone
  than to use guilt or any underhanded psychological tricks that will only engender resentment.

\item[2: Be kind.]
  I don't know if there actually is a book called
  \emph{Secret Tricks of the Ninja Sales Masters},
  but if there is,
  it probably tells readers that doing something for a potential customer
  creates a sense of obligation,
  which in turn increases the odds of a sale.
  That may work, but it only works once and it's a skeezy thing to do.
  On the other hand,
  if you are genuinely kind
  and help other people because it's what good people do,
  you just might inspire them to be good people too.

\item[3: Appeal to the greater good.]
  If you open by talking about what's in it for them,
  you are signaling that they should think of their interaction with you
  as a commercial exchange of value to be bargained over.
  Instead,
  start by explaining how whatever you want them to help with is going to make the world a better place,
  and \emph{mean it}.
  If what you're proposing isn't going to make the world a better place,
  propose something better.

\item[4: Start small.]
  Most people are understandably reluctant to dive into things head-first,
  so give them a chance to test the waters
  and to get to know you and everyone else involved in
  whatever it is you want help with.
  Don't be surprised or disappointed if that's where things end:
  everyone is busy or tired or has projects of their own,
  or maybe just has a different mental model of how collaboration is supposed to work.
  Remember the 90-9-1 rule---90\% of people will watch,
  9\% will speak up,
  and 1\% will actually do things---and set your expectations accordingly.

\item[5: Don't build a project: build a community.]
  I used to belong to a baseball team that never actually played baseball:
  our ``games'' were just an excuse for us to hang out and enjoy each other's company.
  You probably don't want to go quite that far,
  but sharing a cup of tea with someone or celebrating the birth of their first grandchild
  can get you things that no reasonable amount of money can.

\item[6: Establish a point of connection.]
  ``I was speaking to X'' or ``we met at Y'' gives them context,
  which in turn makes them more comfortable.
  This must be specific:
  spammers and cold-callers have trained us all to ignore anything that starts,
  ``I recently came across your website{\ldots}''

\item[7: Be specific about what you are asking for.]
  People need to know this
  so that they can figure out whether the time and skills they have
  are a match for what you need.
  Being realistic up front is also a sign of respect:
  if you tell people you need a hand moving a few boxes
  when you're actually packing up an entire house,
  they're probably not going to help you a second time.

\item[8: Establish your credibility.]
  Mention your backers,
  your size,
  how long your group has been around, or something that you've accomplished in the past
  so that they'll believe you're worth taking seriously.

\item[9: Create a slight sense of urgency.]
  ``We're hoping to launch this in the spring'' is more likely to get a positive response
  than ``We'd eventually like to launch this.''
  However, the word ``slight'' is important:
  if your request is urgent,
  most people will assume you're disorganized or that something has gone wrong
  and may then err on the side of prudence.

\item[10: Take a hint.]
  If the first person you ask for help says no,
  ask someone else.
  If the fifth or the tenth person says no,
  ask yourself if what you're trying to do makes sense and is worth doing.

\end{description}

The email template below follows all of these rules.
It has worked pretty well:
we found that about half of emails were answered,
about half of those wanted to talk more,
and about half of those led to workshops,
which means that 10--15\% of targeted emails turned into workshops.
That can still be pretty demoralizing if you're not used to it,
but is much better than the 2--3\% response rate most organizations expect with cold calls.

\begin{quote}

  \noindent
  Hi NAME,

  I hope you don't mind mail out of the blue,
  but I wanted to follow up on our conversation at VENUE
  to see if you would be interested in having us run a teacher training workshop---we're
  scheduling the next batch over the next couple of weeks.

  This one-day workshop will teach your volunteers
  a handful of practical, evidence-based teaching practices.
  It has been run over a hundred times in various forms on six continents
  for nonprofit organizations, libraries, and companies,
  and all of the material is freely available online at http://teachtogether.tech.
  Topics will include:

  \begin{itemize}
  \item learner personas
  \item differences between different kinds of learners
  \item using formative assessment to diagnose misunderstandings
  \item teaching as a performance art
  \item what motivates and demotivates adult learners
  \item the importance of inclusivity and how to be a good ally
  \end{itemize}
  
  If this sounds interesting,
  please give me a shout---I'd welcome a chance to talk ways and means.

  Thanks,

  NAME

\end{quote}

\begin{aside}{Referrals}
  Building alliances with other groups that are doing things related to your work
  pays off in many ways.
  One of those is referrals:
  if someone who approaches you for help would be better served by some other organization,
  take a moment to make an introduction.
  If you've done this several times,
  add something to your website to help the next person find what they need.
  The organizations you are helping will soon start to help you in return.
\end{aside}

\seclbl{Academic Change}{s:outreach-schools}

Everyone is afraid of the unknown and of embarrassing themselves.
As a result,
most people would rather fail than change.
For example,
Lauren Herckis looked at\index{Herckis, Lauren}
\hreffoot{https://www.insidehighered.com/news/2017/07/06/anthropologist-studies-why-professors-dont-adopt-innovative-teaching-methods}{why university faculty don't adopt better teaching methods}.
She found that the main reason is a fear of looking stupid in front of learners;
secondary reasons were
concern that the inevitable bumps in changing teaching methods would affect course evaluations
(which could in turn affect promotion and tenure)
and people's desire to continue emulating the teachers who had inspired them.
It's pointless to argue about whether these issues are ``real'' or not:
faculty believe they are,
so any plan to work with faculty needs to address them\footnote{
  And the prevalence of fixed mindset among faculty when it comes to teaching,
  i.e.\ the belief that some people are ``just better teachers.''
}.

\cite{Bark2015} did a two-part study of how computer science educators adopt new teaching practices
as individuals, organizationally, and in society as a whole.
They asked and answered three key questions:

\begin{description}

\item[How do faculty hear about new teaching practices?]
  They intentionally seek out new practices
  because they're motivated to solve a problem (particularly student engagement),
  are made aware through deliberate initiatives by their institutions,
  pick them up from colleagues,
  or get them from expected \emph{and unexpected} interactions at conferences
  (teaching-related or otherwise).

\item[Why do they try them out?]
  Sometimes because of institutional incentives
  (e.g.\ they innovate to improve their chances of promotion),
  but there is often tension at research institutions
  where rhetoric about the importance of teaching is largely disbelieved.
  Another important reason is their own cost/benefit analysis:
  will the innovation save them time?
  A third is that they are inspired by role models---again,
  this largely affects innovations aimed to improve engagement and motivation
  rather than learning outcomes---and a fourth is trusted sources,
  e.g.\ people they meet at conferences who are in the same situation they are
  and reported successful adoption.

  But faculty had concerns that were often not addressed by people advocating changes.
  The first was Glass's Law:
  any new tool or practice initially slows you down,
  so while new practices might make teaching more effective in the long run,
  they can't be afforded in the short run.
  Another is that the physical layout of classrooms makes many new practices hard:
  for example,
  discussion groups don't work well in theater-style seating.

  But the most telling result was this:
  ``Despite being researchers themselves,
  the CS faculty we spoke to for the most part did not believe that
  results from educational studies were credible reasons to try out teaching practices.''
  This is consistent with other findings:
  even people whose entire careers are devoted to research often disregard educational research.

\item[Why do they keep using them?]
  As~\cite{Bark2015} says, ``Student feedback is critical,''
  and is often the strongest reason to continue using a practice,
  even though we know that learners' self-reports don't correlate strongly with learning outcomes~\cite{Star2014,Uttl2017}
  (though attendance in lectures is a good indicator of engagement).
  Another reason to retain a practice is institutional requirements,
  although if this is the only motivation,
  people will often drop the practice
  when the explicit incentive or monitoring is removed.

\end{description}

The good news is that you can tackle these problems systematically.
\cite{Baue2015} looked at adoption of new medical techniques within the US Veterans Administration.
They found that evidence-based practices in medicine
take an average of 17 years to be incorporated into routine general practice,
and that only about half of such practices are ever widely adopted.
This depressing finding and others like it spurred the growth of
\gref{g:implementation-science}{implementation science},
which is the study of how to get people to adopt better practices.

As \chapref{s:community} said,
the starting point is to find out what the people you're trying to help believe they need.
For example,
\cite{Yada2016} summarizes feedback from K-12 teachers on the preparation and support they want.
While it may not all be applicable to all settings,
having a cup of tea with a few people and listening before you speak
makes a world of difference to their willingness to try something new.

Once you know what people need,
the next step is to make changes incrementally,
within institutions' own frameworks.
\cite{Nara2018} describes an intensive three-year bachelor's program
based on tight-knit cohorts and administrative support
that tripled graduation rates,
while~\cite{Hu2017} describes the impact of introducing a six-month certification program
for existing high school teachers who want to teach computing.
The number of computing teachers had been stable from 2007 to 2013,
but quadrupled after introduction of the new certification program
without diluting quality:
new-to-computing teachers seemed to be as effective as teachers with more computing training
at teaching the introductory course.

More broadly,
\cite{Borr2014} categorizes ways to make change happen in higher education.
The categories are defined by whether the change is individual or systemic
and whether it is prescribed (top-down) or emergent (bottom-up).
The person trying to make the changes (and make them stick)
has a different role in each situation,
and should pursue different strategies accordingly.
The paper goes on to explain each of the methods in detail,
while~\cite{Hend2015a,Hend2015b} present the same ideas in more actionable form.

Coming from outside,
you will probably fall into the Individual/Emergent category to start with,
since you will be approaching teachers one by one
and trying to make change happen bottom-up.
If this is the case,
the strategies Borrego and Henderson recommend center around
having teachers reflect on their teaching individually or in groups.
Live coding to show them what you do or the examples you use,
then having them live code in turn
to show how they would use those ideas and techniques in their setting,
gives everyone a chance to pick up things that will be useful to them in their context.

\seclbl{Free-Range Teaching}{s:outreach-free-range}

Schools and universities aren't the only places people go to learn programming;
over the past few years, a growing number have turned to free-range workshops
and intensive bootcamp programs.
The latter are typically one to six months long,
run by volunteer groups or by for-profit companies,
and target people who are retraining to get into tech.
Some are very high quality,
but others exist primarily to separate people from their money~\cite{McMi2017}.

\cite{Thay2017} interviewed 26 alumni of such bootcamps
that provide a second chance for those who missed computing education opportunities earlier
(though phrasing it this way makes some pretty big assumptions
when it comes to people from underrepresented groups).
Bootcamp participants face great personal costs and risks:
they must spend significant time, money, and effort before, during, and after bootcamps,
and changing careers can take a year or more.
Several interviewees felt that their certificates were looked down on by employers;
as some said,
getting a job means passing an interview,
but since interviewers often won't share their reasons for rejection,
it's hard to know what to fix or what else to learn.
Many resorted to internships (paid or otherwise)
and spent a lot of time building their portfolios and networking.
The three informal barriers they most clearly identified were jargon,
impostor syndrome,
and a sense of not fitting in.

\cite{Burk2018} dug into this a bit deeper
by comparing the skills and credentials that tech industry recruiters are looking for
to those provided by four-year degrees and bootcamps.
Based on interviews with 15 hiring managers from firms of various sizes and some focus groups,
they found that recruiters uniformly emphasized soft skills
(especially teamwork, communication, and the ability to continue learning).
Many companies required a four-year degree
(though not necessarily in computer science),
but many also praised bootcamp graduates for being older or more mature
and having more up-to-date knowledge.

If you are approaching an existing bootcamp,
your best strategy could be to emphasize what you know about teaching
rather than what you know about tech,
since many of their founders and staff have programming backgrounds
but little or no training in education.
The first few chapters of this book have played well with this audience in the past,
and~\cite{Lang2016} describes
evidence-based teaching practices that can be put in place
with minimal effort and at low cost.
These may not have the most impact,
but scoring a few early wins helps build support for larger efforts.

\seclbl{Final Thoughts}{s:outreach-final}

It is impossible to change large institutions on your own:
you need allies,
and to get allies,
you need tactics.
The most useful guide I have found is~\cite{Mann2015},
which catalogs more than four dozen of these
and organizes them according to whether they're best deployed early,
later,
throughout the change cycle,
or when you encounter resistance.
A handful of their patterns include:

\begin{description}

\item[In Your Space:]
  Keep the new idea visible
  by placing reminders throughout the organization.

\item[Token:]
  To keep a new idea alive in a person's memory,
  hand out tokens that can be identified with the topic being introduced.

\item[Champion Skeptic:]
  Ask strong opinion leaders who are skeptical of your new idea
  to play the role of ``official skeptic.''
  Use their comments to improve your effort,
  even if you don't change their minds.

\item[Future Commitment:]
  If you are able to anticipate some of your needs,
  you can ask for a future commitment from busy people.
  If given some lead time,
  they may be more willing to help.
  
\end{description}

The most important strategy is
to be willing to change your goals
based on what you learn from the people you are trying to help.
Tutorials showing them how to use a spreadsheet
might help them more quickly and more reliably than
an introduction to JavaScript.
I have often made the mistake of confusing things I was passionate about
with things that other people ought to know;
if you truly want to be a partner,
always remember that learning and change have to go both ways.

The hardest part about building relationships is getting started.
Set aside an hour or two every month
to find allies and maintain your relationships with them.
One way to do this is to ask them for advice:
how do they think you ought to raise awareness of what you're doing?
Where have they found space to run classes?
What needs do they think aren't being met
and would you be able to meet them?
Any group that has been around for a few years will have useful advice;
they will also be flattered to be asked,
and will know who you are the next time you call.

And as~\cite{Kuch2011} says,
if you can't be first in a category,
try to create a new category that you can be first in.
If you can't do that,
join an existing group or think about doing something else entirely.
This isn't defeatist:
if someone else is already doing what you have in mind,
you should either chip in or tackle one of the other equally useful things
you could be doing instead.

\seclbl{Exercises}{s:outreach-exercises}

\exercise{Pitching a City Councilor}{individual}{10}

This chapter described an organization
that offers weekend programming workshops for people re-entering the workforce.
Write an elevator pitch for that organization
aimed at a city councilor whose support the organization needs.

\exercise{Pitching Your Organization}{individual}{30}

Identify two groups of people your organization needs support from
and write an elevator pitch aimed at each one.

\exercise{Email Subjects}{pairs}{10}

Write the subject lines (and only the subject lines) for three email messages:
one announcing a new course,
one announcing a new sponsor,
and one announcing a change in project leadership.
Compare your subject lines to a partner's
and see if you can merge the best features of each while also shortening them.

\exercise{Handling Passive Resistance}{small groups}{30}

People who don't want change will sometimes say so out loud,
but may also often use various forms of passive resistance,
such as just not getting around to it over and over again,
or raising one possible problem after another
to make the change seem riskier and more expensive than it's actually likely to be
\cite{Scot1987}.
Working in small groups,
list three or four reasons why people might not want your teaching initiative to go ahead,
and explain what you can do with the time and resources you have to counteract each.

\exercise{Why Learn to Program?}{individual}{15}

Revisit the ``Why Learn to Program?'' exercise in \secref{s:intro-exercises}.
Where do your reasons for teaching and your learners' reasons for learning align?
Where do they not?
How does that affect your marketing?

\exercise{Conversational Programmers}{think-pair-share}{15}

A \gref{g:conversational-programmer}{conversational programmer}
is someone who needs to know enough about computing
to have a meaningful conversation with a programmer,
but isn't going to program themselves.
\cite{Wang2018} found that most learning resources don't address this group's needs.
Working in pairs,
write a pitch for a half-day workshop intended to help people that fit this description
and then share your pair's pitch with the rest of the class.

\exercise{Collaborations}{small groups}{30}

Answer the following questions on your own,
then compare your answers to those given by other members of your group.

\begin{enumerate}

\item
  Do you have any agreements or relationships with other groups?

\item
  Do you want to have relationships with any other groups?

\item
  How would having (or not having) collaborations
  help you to achieve your goals?

\item
  What are your key collaborative relationships?

\item
  Are these the right collaborators for achieving your goals?

\item
  What groups or entities would you like your organization
  to have agreements or relationships with?

\end{enumerate}

\exercise{Educationalization}{whole class}{10}

\cite{Laba2008} explores why the United States and other countries
keep pushing the solution of social problems onto educational institutions
and why that continues not to work.
As he points out,
``[Education] has done very little to promote equality of race, class, and gender;
to enhance public health, economic productivity, and good citizenship;
or to reduce teenage sex, traffic deaths, obesity, and environmental destruction.
In fact,
in many ways it has had a negative effect on these problems
by draining money and energy away from social reforms that might have had a more substantial impact.''
He goes on to write:

\begin{quote}

  So how are we to understand the success of this institution
  in light of its failure to do what we asked of it?
  One way of thinking about this is that
  education may not be doing what we ask,
  but it is doing what we want.
  We want an institution that will pursue our social goals
  in a way that is in line with the individualism at the heart of the liberal ideal,
  aiming to solve social problems
  by seeking to change the hearts, minds, and capacities of individual students.
  Another way of putting this is that
  we want an institution through which we can express our social goals
  without violating the principle of individual choice
  that lies at the center of the social structure,
  even if this comes at the cost of failing to achieve these goals.
  So education can serve as a point of civic pride,
  a showplace for our ideals,
  and a medium for engaging in uplifting but ultimately inconsequential disputes
  about alternative visions of the good life.
  At the same time,
  it can also serve as a convenient whipping boy
  that we can blame for its failure to achieve our highest aspirations for ourselves as a society.

\end{quote}

How do efforts to teach computational thinking and digital citizenship in schools
fit into this framework?
Do bootcamps avoid these traps or simply deliver them in a new guise?

\exercise{Institutional Adoption}{whole class}{15}

Re-read the list of motivations to adopt new practices
given in \secref{s:outreach-schools}.
Which of these apply to you and your colleagues?
Which are irrelevant to your context?
Which do you emphasize
if and when you interact with people working in formal educational institutions?

\exercise{If At First You Don't Succeed}{small groups}{15}

W.C.~Fields probably never said,
``If at first you don’t succeed, try, try again.
Then quit---there's no use being a damn fool about it.''
It's still good advice:
if the people you're trying to reach aren't responding,
it could be that you'll never convince them.
In groups of 3--4,
make a short list of signs that you should stop trying to do something you believe in.
How many of them are already true?

\exercise{Making It Fail}{individual}{15}

\cite{Farm2006} presents some tongue-in-cheek rules for ensuring that new tools \emph{aren't} adopted,
all of which apply to new teaching practices:

\begin{enumerate}

\item
  Make it optional.

\item
  Economize on training.

\item
  Don't use it in a real project.

\item
  Never integrate it.

\item
  Use it sporadically.

\item
  Make it part of a quality initiative.

\item
  Marginalize the champion.

\item
  Capitalize on early missteps.

\item
  Make a small investment.

\item
  Exploit fear, uncertainty, doubt, laziness, and inertia.

\end{enumerate}

Which of these have you seen done recently?
Which have you done yourself?
What form did they take?

\exercise{Mentoring}{whole class}{15}

The \hreffoot{http://www.iaamcs.org/}{Institute for African-American Mentoring in Computer Science}
has published \hreffoot{http://iaamcs.org/guidelines}{guidelines for mentoring doctoral students}.
Read them individually,
then go through them as a class
and rate your efforts for your own group as +1 (definitely doing),
0 (not sure or not applicable),
or -1 (definitely not doing).
