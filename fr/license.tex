\chapter{License}\label{s:license}

{\setlength{\parindent}{0em}

\emph{
  This is a human-readable summary of (and not a substitute for) the license.
  Please see \url{https://creativecommons.org/licenses/by-nc/4.0/legalcode} for the full legal text.
}

\vspace{\baselineskip}

\noindent
This work is licensed under the
\hreffoot{https://creativecommons.org/licenses/by-nc/4.0/}{Creative Commons Attribution-NonCommercial 4.0} license
(CC-BY-NC-4.0).\\

\noindent
\textbf{You are free to:}

\begin{itemize}
\item
  \textbf{Share}---copy and redistribute the material in any medium or
  format
\item
  \textbf{Adapt}---remix, transform, and build upon the material.
\end{itemize}

The licensor cannot revoke these freedoms as long as you follow the
license terms.

\vspace{\baselineskip}

\textbf{Under the following terms:}

\begin{itemize}
\item
  \textbf{Attribution}---You must give appropriate credit, provide a link
  to the license, and indicate if changes were made. You may do so in
  any reasonable manner, but not in any way that suggests the licensor
  endorses you or your use. \\
\item
  \textbf{NonCommercial}---You may not use the material for commercial purposes.
\end{itemize}

\textbf{No additional restrictions}---You may not apply legal terms or
technological measures that legally restrict others from doing anything the
license permits.

\vspace{\baselineskip}

\textbf{Notices:}

\begin{itemize}

\item
  You do not have to comply with the license for elements of the
  material in the public domain or where your use is permitted by an
  applicable exception or limitation.

\item
  No warranties are given. The license may not give you all of the
  permissions necessary for your intended use. For example, other rights
  such as publicity, privacy, or moral rights may limit how you use the
  material.

\end{itemize}
}
